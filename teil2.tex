Bedingung: $$V\subseteq X^*\backslash W\cdot X^*$$

\begin{eigen}
$$V\ff W\cdot X^* = V\ff W $$
\end{eigen}

Es gen�gt die Eigenschaft zu zeigen f�r $L=\{v\}$:
\begin{proof}
$$V\ff W\cdot X^* = \{v\} \ff W\cdot X^*$$
Nach Definition der Operation Fortsetzung ergibt sich:
$$ = min_{\sqsubseteq} \{w:w\in W\cdot X^* \wedge v\sqsubseteq w\}$$
Wir wissen aus der Vorrausetzung, dass $v\notin W\cdot X^*$, daraus folgt unmittelbar, dass $v\in pref(W)\backslash W$.\\
Angenommen es existiert ein $w\in W\cdot X^+$ in $min_{\sqsubseteq} \{w:w\in W\cdot X^* \wedge v\sqsubseteq w\}$, so ist $w'$ mit $w=w'\cdot r\wedge w'\in W$ ein k�rzeres Wort, daher gilt:
$$min_{\sqsubseteq} \{w:w\in W\cdot X^* \wedge v\sqsubseteq w\} = min_{\sqsubseteq} \{w:w\in W \wedge v\sqsubseteq w\}$$
\end{proof}

\begin{eigen}
$$V\ff W = V\ff \min(W)$$
\end{eigen}

\newpage

\begin{eigen}
$\pref(V)\ff W$
\end{eigen}
\begin{proof}
1. Fall: $w\in W \cap \pref(V) \to w\in \pref(V)\ff W$\\
2. Fall: $w\in W\backslash \pref(V) \to w\in V\ff Min(W)$\\
$\to \pref(V)\ff W = (\pref(V)\cap W) \cup (\pref(V)\ff \min(W))$\\\\
\end{proof}

\begin{eigen}
$W\ff \pref(V) = W \cap \pref(V)$
\end{eigen}
Es gen�gt die Eigenschaft f�r $W=\{w\}$ zu zeigen:
\begin{proof}
$$W\ff \pref(V) = \{w\}\ff\pref(V)$$
$$=min_{\sqsubseteq} \{v:v\in \pref(V) \wedge w\sqsubseteq v\}$$
Aus $v\in \pref(V) \wedge w\sqsubseteq v$ folgt direkt, dass $w\in pref(V)$. Damit ergibt sich :
$$min_{\sqsubseteq} \{v:v\in \pref(V) \wedge w\sqsubseteq v\} = \{v:v\in\pref(V)\wedge w=v\} = \{w\} \cap \pref(V)$$
\end{proof}
%"'$\leftarrow$"'\\
%$w\in \pref(V) \wedge w\in W \to w\sqsubseteq v\in \pref(V)$\\\\
\begin{eigen}
$V\cdot X^* \ff W$
\end{eigen}
\begin{proof}
$\bigcup_{v\in VX^*} \min_{\sqsubseteq} \{w:w\in W \wedge v\sqsubseteq w\}$\\
$v\in VX^* \wedge v\sqsubseteq w \to w\in VX^*$\\
$=\{w:w\in W \wedge w\in VX^*\} = W\cap VX^*$\\\\
%$=\bigcup_{v\in V} \{w:w\in W \wedge v\sqsubseteq w\}$\\\\
\end{proof}

\begin{eigen}
$$W\ff V\cdot X^*$$
\end{eigen}
\begin{proof}
Die Aussage l�sst sich in 2 F�lle aufteilen:\\\\
Fall a) $W\subseteq X^* \backslash V\cdot X^*$\\
Fall b) $W\subseteq V\cdot X^*$\\\\
Es gen�gt die Eigenschaft zu zeigen f�r $W=\{w\}$.\\
zu a): Wie in Eigenschaft 3.2 und 3.3 bereits gezeigt folgt aus der der Bedingung in Fall a): $W\subseteq X^* \backslash V\cdot X^*\to W\ff V\cdot X^* = W\ff\min(V)$\\
zu b): zu betrachten $\{w\}\ff V\cdot X^*$
$$\{w\}\ff V\cdot X^* = min_{\sqsubseteq} \{ v:v\in V\cdot X^* \wedge w\sqsubseteq  \}$$
Da wir aus der Bedingung von Fall b) wissen, dass $w\in V\cdot X^*$, so folgt aus:\\
$min_{\sqsubseteq} \{ v:v\in V\cdot X^* \wedge w\sqsubseteq v\}$ unmittelbar, dass es sich um die Menge\\ $\{v:v\in V\cdot X^* \wedge w=v\} = \{w\} \cap V\cdot X^*$ handelt.\\
\\Daraus folgt direkt: $$W\ff V\cdot X^* = (W\ff\min(V)) \cup (W\cap V\cdot X^*)$$
\end{proof}





%\newpage
%$$ V = \{v:\forall w(w\in W \to w\not\sqsubseteq v)\}\qquad \mathit{pref}(V)=\{u:\exists v (u\sqsubseteq v\wedge v\in V)\}$$\\
%$$W\ff V = \emptyset$$
%$$W\ff \mathit{pref}(V) = \emptyset$$
%$$W\cdot X^* \ff V = \emptyset$$
%$$W\cdot X^* \ff \mathit{pref}(V) = \emptyset$$
%$$\mathit{pref}(V)\ff W\cdot X^* = $$

%$$W\ff V = \bigcup_{x\in W} x\ff V = \bigcup \min_{\sqsubseteq } \{v:v\in V \wedge x\sqsubseteq v\}$$
%Wegen $x\in W$ und $x\sqsubseteq v$ gilt $\{v:v\in V \wedge x\sqsubseteq v\} = \emptyset \Rightarrow \bigcup \min_{\sqsubseteq} \{v:v\in V \wedge x\sqsubseteq v\} = \emptyset$\\\\
%$$ W\cdot X^* \ff V = \bigcup_{x\in W\cdot X^*} x\ff V = \bigcup_{x\in W\cdot X^*} \min_{\sqsubseteq} \{v:v\in V \wedge x\sqsubseteq v\} = \emptyset$$
%Begr�ndung analog Fall $W\ff V$\\\\
%$$ W \ff \mathit{pref}(V) = \bigcup_{w\in W} w\ff \mathit{pref}(V) = \bigcup_{w\in W} \{v:v\in \mathit{pref}(V) \wedge w\sqsubseteq v\} = \emptyset$$
%$w\sqsubseteq v$ ist nie erfuellt. Angenommen $v\in \mathit{pref}(V) \wedge w\sqsubseteq v$, dann kann man $v$ wie folgt verlaengern $v'=v\cdot v''$ mit $v'\in V$, dann waere $v'=w\cdot w' \cdot v''$ mit $w\cdot w' = v$.
%Dadurch gilt aber $v'\in V\wedge w\sqsubseteq v' \wedge w\in W \Rightarrow$ Widerspruch zur Definition.\\\\

%$$\mathit{pref}(V) \ff W\cdot X^* = \bigcup_{u\in \mathit{pref}(V)} u\ff W\cdot X^* = \bigcup_{u\in \mathit{pref}(V)} \min_{\sqsubseteq} \{w:w\in W\cdot X^* \wedge u\sqsubseteq w\}$$
%$$ =  \bigcup_{u\in \mathit{pref}(V)} \min(W\cdot X^* \cap u\cdot X^*) =  \bigcup_{u\in \mathit{pref}(V)} \min(\{w:w\in W \wedge u\sqsubseteq w\})$$\\
%$$W\cdot X^* \ff \mathit{pref}(V) = \bigcup_{u\in W\cdot X^*} u\ff \mathit{pref}(V) = \bigcup_{u\in W\cdot X^*} \min_{\sqsubseteq} \{w:w\in \mathit{pref}(V) \wedge u\sqsubseteq w\}$$
%$$ = \bigcup_{u\in W\cdot X^*} \min(\mathit{pref}(V)\cap u\cdot X^*)$$
%$\mathit{pref}(V)\cap u\cdot X^* = \emptyset$, Begr�ndung siehe oben $\Rightarrow  \bigcup_{u\in W\cdot X^*} \min(\emptyset) = \emptyset$
