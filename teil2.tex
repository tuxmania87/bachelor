%\begin{eigen}
%$$V\ff W\cdot X^* = V\ff W $$
%\end{eigen}

%Es gen"ugt die Eigenschaft zu zeigen f"ur $L=\{v\}$:
%\begin{proof}
%$$V\ff W\cdot X^* = \{v\} \ff W\cdot X^*$$
%Nach Definition der Operation Fortsetzung ergibt sich:
%$$ = min_{\sqsubseteq} \{w:w\in W\cdot X^* \wedge v\sqsubseteq w\}$$
%Wir wissen aus der Vorrausetzung, dass $v\notin W\cdot X^*$, daraus folgt unmittelbar, dass $v\in pref(W)\setminus W$.\\
%Angenommen es existiert ein $w\in W\cdot X^+$ in $min_{\sqsubseteq} \{w:w\in W\cdot X^* \wedge v\sqsubseteq w\}$, so ist $w'$ mit $w=w'\cdot r\wedge w'\in W$ ein k"urzeres Wort, daher gilt:
%$$min_{\sqsubseteq} \{w:w\in W\cdot X^* \wedge v\sqsubseteq w\} = min_{\sqsubseteq} \{w:w\in W \wedge v\sqsubseteq w\}$$
%\end{proof}

%\begin{eigen}
%$$V\ff W = V\ff \min(W)$$
%\end{eigen}

In der Arbeit [St97, Abschnitt: \emph{Joint topologies on $X^*\cup X^\omega$}] wird eine Sprache $W$ als offen definiert genau dann, wenn sie die Form $W=W\cdot X^*$ hat. Weiterhin wird eine Sprache $L$ als geschlossen definiert genau dann, wenn sie die Form $L=\pref(L)$ hat. 
Daher betrachten wir in diesem Abschnitt die Operation Fortsetzung, wenn eine der beiden Operanden die spezielle Gestalt $\pref(L)$ oder $W\cdot X^*$ hat. 

\vspace{2ex}

\begin{eigen}\label{spezeig1}
$W\ff \pref(V) = W \cap \pref(V)$
\end{eigen}
\begin{proof}
Die Inklusion $\supseteq$ folgt aus Eigenschaft \ref{klaro}.\\
Zum Beweis der anderen Inklusion betrachten wir $w\in W\ff\pref(V)$. Es muss also ein $u\in W$ existieren, sodass $w\in \{u\}\ff\pref(V)$. Daraus folgt, dass $u$ ein Pr"afix von $w$ sein muss, was wiederum bedeutet, dass $u\in\pref(V)$. Damit haben wir $w\in\min(\{u\}\cdot X^*\cap \pref(V)) \to w\in \{u\}\cap \pref(V)$.
%Es gen"ugt die Eigenschaft f"ur $W=\{w\}$ zu zeigen:\\
%Aus der Definition wissen wir, dass $w\ff \pref(V)= \min ( \{w\} \cdot X^* \cap \pref(V) )$ gilt. 
%ist $w'\in (\{w\} \cdot X^* \cap \pref(V))$, so sieht man, dass $w\sqsubseteq w'$ gilt. Also haben wir wegen $w'\in \pref(V)$ auch $w\in\pref(V)$.
%Daraus folgt $\min ( \{w\} \cdot X^* \cap \pref(V) ) = \{w\} \cap \pref(V)$
\end{proof}


Betrachten wir nun die Umkehrung von Eigenschaft \ref{spezeig1}: $\pref(V)\ff W$. Wir k"onnen mit Eigenschaft \ref{basiseigen} die Fortsetzung $\pref(V)\ff W$ aufspalten in $\pref(V)\ff W= (\pref(V)\cap W) \cup ((pref(V)\setminus W)\ff W)$, aber diese Gleichung lässt sich nicht weiter vereinfachen.

%\begin{eigen}
%%$W\ff \pref(V) = W \cap \pref(V)$%
%\end{eigen}
%\begin{proof}
%Es gen"ugt die Eigenschaft f"ur $W=\{w\}$ zu zeigen:
%$$W\ff \pref(V) = \{w\}\ff\pref(V)$$
%$$=min_{\sqsubseteq} \{v:v\in \pref(V) \wedge w\sqsubseteq v\}$$
%Aus $v\in \pref(V) \wedge w\sqsubseteq v$ folgt direkt, dass $w\in pref(V)$. Damit ergibt sich :
%$$min_{\sqsubseteq} \{v:v\in \pref(V) \wedge w\sqsubseteq v\} = \{v:v\in\pref(V)\wedge w=v\} = \{w\} \cap \pref(V)$$
%\end{proof}
%"'$\leftarrow$"'\\
%$w\in \pref(V) \wedge w\in W \to w\sqsubseteq v\in \pref(V)$\\\\

\vspace{2ex}

\begin{eigen}
$V\cdot X^* \ff W = V\cdot X^* \cap W$
\end{eigen}
\begin{proof}
Die Inklusion $\supseteq$ folgt aus Eigenschaft \ref{klaro}.\\
Zum Beweis der anderen Inklusion betrachten wir $w\in V\cdot X^* \ff W$. Es existiert also ein $v\in V\cdot X^*$, sodass $w\in \{v\}\ff W$ ist. Daraus folgt, dass $v$ ein Pr"afix von $w$ sein muss, was wiederum bedeutet, dass $w\in V\cdot X^*$. Somit gilt die Inklusion $V\cdot X^* \ff W \subseteq W\cap V\cdot X^*$.
\end{proof}

%\begin{eigen}
%$V\cdot X^* \ff W$
%\end{eigen}
%\begin{proof}
%Es gen"ugt die Eigenschaft f"ur ein $v\in V\cdot X^*$ zu zeigen:
%$$\min_{\sqsubseteq} \{w:w\in W \wedge v\sqsubseteq w\}$$
%$$v\in VX^* \wedge v\sqsubseteq w \to w\in VX^*$$
%$$=\min_{\sqsubseteq} \{w:w\in W \wedge w\in V\cdot X^* \wedge v=w\} = \{v\}\cap W$$
%$=\bigcup_{v\in V} \{w:w\in W \wedge v\sqsubseteq w\}$\\\\
%\end{proof}

\begin{eigen}\label{fooba}
$W\ff V\cdot X^* = (W\ff\min(V)) \cup (W\cap V\cdot X^*)$
\end{eigen}
Zum Beweis von Eigenschaft \ref{fooba} benutzen wir

\vspace{2ex}

\begin{lem}\label{eingeschr1}
Es sei $L\subseteq X^*\setminus W\cdot X^*$, so gilt $L\ff W\cdot X^* = L\ff \min(W) $
\end{lem}
\begin{proof}[Beweis Lemma \ref{eingeschr1}]:\\
Es gilt für alle W"orter $w\in\min(W)$, dass kein Wort $w'\in W\cdot X^*$ ein echtes Präfix von $w$ sein kann.
Daraus folgt unmittelbar die Inklusion $L\ff \min(W) \subseteq L\ff W\cdot X^*$.
Zum Beweis der anderen Inklusion betrachten wir ein $w\in L\ff W\cdot X^*$. Es muss also ein $v\in L$ existieren, sodass $w\in \{v\}\ff W\cdot X^*$ und, nach Vorraussetzung, $v\notin W\cdot X^*$. Dann gilt f"ur kein $u$, $v\sqsubseteq u\sqsubset w$ oder $u\sqsubseteq v$ die Beziehung $u\in W\cdot X^*$.
Also haben wir $w\in\min(W\cdot X^*) = \min(W)$ und somit gilt auch die Inklusion $L\ff \min(W) \supseteq L\ff W\cdot X^*$.
\end{proof}

\begin{proof}[Beweis von Eigenschaft \ref{fooba}]
Die Fortsetzung von $W$ in $V\cdot X^*$ l"asst sich, mit Hilfe von Eigenschaft \ref{basiseigen}, in zwei Teile aufspalten (\ref{arr1}).
Da $W\setminus V\cdot X^* \subseteq X^*\setminus V\cdot X^*$, wenden wir Lemma \ref{eingeschr1} an und erhalten (\ref{arr3}), was sich zu (\ref{arr4}) vereinfachen l"asst.
\setcounter{equation}{0}
\begin{eqnarray}
W\ff V\cdot X^* & = & ((W\setminus V\cdot X^*) \ff V\cdot X^*) \cup (W\cap V\cdot X^*) \label{arr1}\\
%V' &:= & W\setminus V\cdot X^* \label{arr2}\\
W\ff V\cdot X^* & = & ((W\setminus V\cdot X^*)  \ff \min(V\cdot X^*) \cup (W\cap V\cdot X^*) \label{arr3} \\
W\ff V\cdot X^* & = &(W \ff \min(V)) \cup (W\cap V\cdot X^*) \label{arr4}
\end{eqnarray}
\end{proof}





%\newpage
%$$ V = \{v:\forall w(w\in W \to w\not\sqsubseteq v)\}\qquad \mathit{pref}(V)=\{u:\exists v (u\sqsubseteq v\wedge v\in V)\}$$\\
%$$W\ff V = \emptyset$$
%$$W\ff \mathit{pref}(V) = \emptyset$$
%$$W\cdot X^* \ff V = \emptyset$$
%$$W\cdot X^* \ff \mathit{pref}(V) = \emptyset$$
%$$\mathit{pref}(V)\ff W\cdot X^* = $$

%$$W\ff V = \bigcup_{x\in W} x\ff V = \bigcup \min_{\sqsubseteq } \{v:v\in V \wedge x\sqsubseteq v\}$$
%Wegen $x\in W$ und $x\sqsubseteq v$ gilt $\{v:v\in V \wedge x\sqsubseteq v\} = \emptyset \Rightarrow \bigcup \min_{\sqsubseteq} \{v:v\in V \wedge x\sqsubseteq v\} = \emptyset$\\\\
%$$ W\cdot X^* \ff V = \bigcup_{x\in W\cdot X^*} x\ff V = \bigcup_{x\in W\cdot X^*} \min_{\sqsubseteq} \{v:v\in V \wedge x\sqsubseteq v\} = \emptyset$$
%Begr"undung analog Fall $W\ff V$\\\\
%$$ W \ff \mathit{pref}(V) = \bigcup_{w\in W} w\ff \mathit{pref}(V) = \bigcup_{w\in W} \{v:v\in \mathit{pref}(V) \wedge w\sqsubseteq v\} = \emptyset$$
%$w\sqsubseteq v$ ist nie erfuellt. Angenommen $v\in \mathit{pref}(V) \wedge w\sqsubseteq v$, dann kann man $v$ wie folgt verlaengern $v'=v\cdot v''$ mit $v'\in V$, dann waere $v'=w\cdot w' \cdot v''$ mit $w\cdot w' = v$.
%Dadurch gilt aber $v'\in V\wedge w\sqsubseteq v' \wedge w\in W \Rightarrow$ Widerspruch zur Definition.\\\\

%$$\mathit{pref}(V) \ff W\cdot X^* = \bigcup_{u\in \mathit{pref}(V)} u\ff W\cdot X^* = \bigcup_{u\in \mathit{pref}(V)} \min_{\sqsubseteq} \{w:w\in W\cdot X^* \wedge u\sqsubseteq w\}$$
%$$ =  \bigcup_{u\in \mathit{pref}(V)} \min(W\cdot X^* \cap u\cdot X^*) =  \bigcup_{u\in \mathit{pref}(V)} \min(\{w:w\in W \wedge u\sqsubseteq w\})$$\\
%$$W\cdot X^* \ff \mathit{pref}(V) = \bigcup_{u\in W\cdot X^*} u\ff \mathit{pref}(V) = \bigcup_{u\in W\cdot X^*} \min_{\sqsubseteq} \{w:w\in \mathit{pref}(V) \wedge u\sqsubseteq w\}$$
%$$ = \bigcup_{u\in W\cdot X^*} \min(\mathit{pref}(V)\cap u\cdot X^*)$$
%$\mathit{pref}(V)\cap u\cdot X^* = \emptyset$, Begr"undung siehe oben $\Rightarrow  \bigcup_{u\in W\cdot X^*} \min(\emptyset) = \emptyset$
