%notationen
Mit $\mathbb{N}=\{0,1,2,3,..\}$ bezeichnen wir die Menge der nat�rlichen Zahlen. Sei $X$ ein Alphabet.
Mit $X^*$ bezeichnen wir die Menge aller endlichen W�rter �ber dem Alphabet $X$, einschlie�lich des leeren Wortes $e$.
Wir bezeichnen weiterhin die Menge $X^\omega$ als Menge aller unendlichen W�rter �ber dem Alphabet $X$.\\\\
F�r $w\in X^*$ und $v\in X^\omega$ bezeichnen wir $w\cdot v$ als deren Konkatenation. F�r eine Sprache $W$ ist $W^*=\bigcup_{i\in\mathbb{N}} W^i$. Weiterhin bezeichnet $\vert w\vert$ die L�nge des Wortes $w\in X^*$.
Wir definieren die Pr�fixrelation $\sqsubseteq$ wie �blich mit $w\sqsubseteq b \Leftrightarrow w\cdot b' = b,\textit{ f{\"u}r ein }b'\in X^*$.
Somit bildet $\pref(L) = \{v:v\sqsubseteq w \wedge w\in L\}$ alle Pr�fixe aller W�rter aus $L$.\\\\
Weiterhin bezeichnen wir mit $\min(L) = \{w:w\in L \wedge \forall v( v\sqsubset w \to v\notin L)\}$ alle W�rter der Sprache $L$, in denen kein Wort Pr�fix eines anderen Wortes ist.
Abschlie�end bezeichnen wir die Operation Fortsetzung einer Sprache $L$ in die Sprache $W$ als $L\ff W = \bigcup_{u\in L} \min(u\cdot X^* \cap W)$\\\\
Wir definieren den $\delta$-Limes einer Wortmenge $W ^\delta$ wie in [St87] mit\\$W^\delta = \{\beta : \beta \in X^\omega \text{ und }\pref(\beta)\cap W \text{ ist unendlich}\}$
% also eine unendliche Aneinanderreihung von W�rtern aus ....
Die Eigenschaft (13) aus [St87] ist leich einzusehen: $(U \cup W)^\delta = U^\delta \cup W^\delta$.	Eine Sprache nennen wir eine $(\sigma,\delta)$-Teilmenge von $X^*$ genau dann, wenn f�r alle $\beta \in X^\omega$ entweder $\pref(\beta)\cap W$ oder $\pref(\beta)\backslash W$ endlich ist.
Beispiele f�r $(\sigma,\delta)$-Teilmengen sind alle endlichen Sprachen und deren Komplemente. Weitere Beispiele sind Sprachen der Form $\pref(L)$ oder $W\cdot X^*$.\\
Eine Eigenschaft f�r diese Teilmengen ist:
\begin{satz}[St87]
$\text{Sei }U\text{ eine }(\sigma,\delta)-\text{Teilmenge von }X^*, \text{ dann gilt:}$\\
$(U\cap W)^\delta = U^\delta \cap W^\delta,\qquad \text{f�r alle }W\subseteq X^*$
\end{satz}
Die Operation Fortsetzung hat nun folgende Eigenschaft bez�glich des $\delta$-Limes.
W�hrend $(W\cap U)^\delta = W^\delta \cap U^\delta$ nur f�r $(\sigma,\delta)$-Teilemgen gilt, so gilt
$(W\ff U)^\delta = W^\delta \cap U^\delta$ f�r s�mtliche Sprachen.\\



