In diesem Abschnitt untersuchen wir die Abgeschlossenheitseigenschaften der Klassen der CHOMSKY-Hierachie bez"uglich der Operation Fortsetzung. 
Dabei betrachten wir in den folgenden Abschnitten die Klassen der regul"aren, kontextfreien, aufz"ahlbaren und entscheidbaren Sprachen.

\section{Regularit"at}\label{abschnittreg}
\begin{satz}
Sind $L$ und $W$ regul\"are Sprachen, so ist auch $L\ff W$ regul\"ar.
\end{satz}
\begin{proof}
Der nicht-deterministische Automat $A_L= (X,Z,z_{0},\delta_L,Z_f)$ akzeptiere $L$, der deterministische Automat $A_W = (X,S,s_{0},f,S_{f})$ akzeptiere $W$.
Wir konstruieren einen nicht-determninistischen Automaten $A$, der $L\ff W$ akzeptiert.\\\\\emph{Arbeitsweise.}\\
In Phase eins lesen $A_L$ und $A_W$ das Wort $w$ parallel (\ref{auto1}). Falls $A_L$ ein Pr"afix $v$ von $w$ akzeptiert, w"ahlt $A$ nicht-deterministisch aus, ob Schritt zwei aktiviert wird oder nicht ( Nicht-Determinismus von $A_L$ ).\\
Wird Phase zwei aktiviert, so liest $A_W$ das Wort $w$ weiter, w"ahrend $A_L$ in einem Ruhezustand $z_f'$ verweilt (\ref{auto2}). Sobald $A_W$ in Phase zwei akzeptiert, gelangt $A$, nach Konstruktion, in den Zustand $(z_f',s_x)$ aus dem keine Transition herausf"uhrt (\ref{auto4}). Akzeptiert $A_W$ nun ein Pr"afix $v'$ mit $v\sqsubseteq v'\sqsubset w\quad v'\in W$, so muss $A$ das Eingabewort $w$ ablehnen. Dies wird realsiert, da keine Transition aus $(z_f',s_x)$ herausf"uhrt und $w$ noch nicht zu Ende gelesen ist. Findet der Automat $A_W$ kein solches Pr"afix $v'$ und akzeptiert das Eingabewort $w$ ( und kein Pr"afix von $w$), so akzeptiert auch $A$, weil $w$ zu Ende gelesen wurde und sich $A$ im Finalzustand $(z_f',s_x)$ (\ref{auto4}) befindet.
\\$A = (X,(Z\cup \{ z_f'\})\times (S\cup \{ s_x\}), (z_{0},s_{0}), \delta , \{ (z_f',s_x) \} ),$ mit $s_x\notin S,\ z_f'\notin Z$
\setcounter{equation}{0}
\begin{eqnarray}
 \delta &=&\{ ((z_i,s_i),x,(z_j,s_j)) : (z_i,x,z_j) \in \delta_L \wedge f(s_i,x)=s_j  \}  \label{auto1} \\
 & \cup & \{ ((z_i,s_i),x,(z_f',s_j)) : (z_i,x,z') \in \delta_L \wedge z'\in Z_f \wedge f(s_i,x)=s_j  \} \label{auto2}\\
 & \cup & \{ ((z_f',s_i),x,(z_f',s_j)) : f(s_i,x)=s_j  \wedge s_j \notin S_f\} \label{auto3}\\
 & \cup & \{ ((z_f',s_i),x,(z_f',s_x)) : f(s_i,x)=s_j  \wedge s_j \in S_f \label{auto4} \}
\end{eqnarray}
Nach Konstruktion ist klar, dass der Automat $A$ nur W"orter aus $L\ff W$ akzeptiert. Falls $w\in L\ff W$, so gibt es ein Präfix $v\in L$, sodass $w\in \{v\}\ff W$. Weil dieses Präfix nicht-deterministisch von $A_L$ ausgew"ahlt wird, akzeptiert $A$ auch alle W"orter aus $L\ff W$.
Damit akzeptiert der Automat $A$ ein Eingabewort $w$ genau dann, wenn $w\in L\ff W$.
\end{proof}

\section{Kontextfreiheit}
Es existieren deterministisch kontextfreie, lineare Sprachen $L,W$ derart, dass $L\ff W$ nicht einmal kontextfrei ist.

\vspace{2ex}

\begin{beispiel}
Es seien $L=\{a^nb^nc^i:i,n>0\}$ und $W=\{a^ib^nc^n:i,n>0\}$. Sowohl $L$ als auch $W$ sind deterministische, kontextfreie und auch lineare Sprachen, da es je einen deterministischen Kellerautomaten gibt, der $L$ sowie $W$ akzeptiert und es lineare Grammatiken \\\\$G_L = (\{S,A,C\},\{a,b,c\},S,\{ (S,Cc),(C,Cc),(C,aAb),(A,aAb),(A,e)\})$ und \\$G_W = (\{S,A,B\},\{a,b,c\},S,\{ (S,aA),(A,aA),(A,bBc),(B,bBc),(B,e) \})$ gibt, sodass \\$L(G_L) = L$ und $L(G_W) = W$ gilt.\\\\
Betrachten wir nun ein Wort $u\in L\ff W$ so muss $u$ laut Definition folgende Struktur besitzen $u\in \min( v\cdot X^* \cap W)$ f"ur ein $v \in L$.
Damit sieht man leicht, dass $L\ff W= \{a^nb^nc^n:n>0\}$ gilt. Diese Sprache ist bekanntlich nicht einmal kontextfrei.
\end{beispiel}

\vspace{2ex}

\begin{satz}
Ist $L$ eine kontextfreie Sprache und $W$ eine reguläre Sprache, so ist $L\ff W$ ebenfalls kontextfrei.
\end{satz}
Zum Beweis könnten wir die Konstruktion aus Abschnitt \emph{\ref{abschnittreg} Regularit"at} verwenden, indem wir den nicht-deterministischen endlichen Automaten $A_L$ durch einen nicht-deterministischen Kellerautomaten ersetzen.
Ein anderer Beweis für diesen Satz wurde in [St97, Abschnitt \emph{Limit-closure}] geführt.
\newpage
\section{Entscheidbarkeit und Aufz"ahlbarkeit}
In diesem Abschnitt betrachten wir, ob f"ur entscheidbare bzw. aufz"ahlbare Sprachen $L$ und $W$ auch deren Fortsetzung $L\ff W$ wiederum entscheidbar bzw. aufz"ahlbar ist. Dabei betrachten wir drei verschieden F"alle.\\\\
Wir betrachten zun"achst $L\ff W$, wenn $L$ und $W$ entscheidbar sind. Dabei werden wir feststellen, dass auch die Fortsetzung $L\ff W$ entscheidbar ist. Ein Algorithmus wird dazu angegeben.\\
Anschlie"send werden wir die Fortsetzung betrachten, wenn $L$ nur aufz"ahlbar und $W$ entscheidbar ist. In diesem Fall ist die Fortsetzung von $L$ in $W$ ebenso aufz"ahlbar. Dazu geben wir einen Algorithmus an.\\
Im letzten Fall betrachten wir $L\ff W$, wenn $W$ nur aufz"ahlbar ist. In diesem Fall ist die Fortsetzung beider Sprachen nicht einmal aufz"ahlbar. Dies gilt sogar, wenn $L$ eine endliche Sprache ist.
\subsection{L und W entscheidbar}

\begin{satz}\label{entsch1}
Sind $L$ und $W$ entscheidbare Sprachen, so ist auch $L\ff W$ entscheidbar.
\end{satz}
\begin{proof}
Wir wissen, dass zu jeder entscheidbaren Sprache $L$ ein Algorithmus angegeben werden kann, welcher $L$ entscheidet. Daher geben wir zum Beweis von Satz \ref{entsch1} einen Algorithmus an, welcher $L\ff W$ entscheidet (befindet sich auf n"achster Seite).\\\\
\emph{Arbeitsweise des Algorithmus:}\\
Zun"achst pr"ufen wir, ob $w\in W$, da dies eine zwingende Voraussetzung nach Definition ist (\# 1).
Jetzt wissen wir, dass $w\in W$. Gilt zus"atzlich $w\in L$, so ist klar, dass $w\in L\ff W$ nach Eigenschaft \ref{klaro} (\# 2).\\\\
Jetzt setzen wir $w'$ mit $w':=w$ als Arbeitskopie. Im folgenden Schleifendurchlauf schneiden wir mit $cut()$ den letzten Buchstaben von $w'$ ab (\# 3). 
Gilt jetzt $w'\in W$ so m"ussen wir mit NEIN entscheiden, da das Präfix $w'\in W$ im Widerspruch zum $\min$ Kriterium steht.
Anschlie"send pr"ufen wir, ob $w'\in L\setminus W$ (\# 5) gilt. Wenn ja, dann haben wir das erste Pr"afix von $w$ gefunden, welches nicht in $W$ liegt, und k"onnen f"ur die Eingabe $w$ mit JA entscheiden.
\\\\Ansonsten durchlaufen wir die Schleife solange weiter, bis $w'$ nur noch aus dem leeren Wort besteht, anschliessend lehnen wir ab, da wir kein Pr"afix von $v$ von $w$ gefunden haben, welches in $L\setminus W$ liegt.
\begin{algorithm}
\caption{entscheide $L\ff W$}
\label{split}
\begin{algorithmic}
%\REQUIRE Menge X, int stufe
%Input: w
\STATE Input $w$
\IF{($w \notin W$)}  
\STATE $T$ rejects \COMMENT{\# 1}
\ELSE
\IF{($w \in L$)}
\STATE $T$ accepts  \COMMENT{\# 2}
\ENDIF
\ENDIF
\STATE $w' = w$  \COMMENT{\# 3}
\REPEAT
\STATE $w' \leftarrow \mathit{cut}(w')$ \COMMENT{\# 4}
\IF{($w'\in W$)}
\STATE $T$ rejects
\ENDIF
\COMMENT{\# 5}
\IF{($w' \in L$)}
\STATE $T$ accepts
\ENDIF
\UNTIL{($w'==e$)}
\STATE $T$ rejects
\end{algorithmic}
\end{algorithm}

\end{proof}


\subsection{L aufz"ahlbar, W entscheidbar}
\begin{satz}
Ist $L$ eine aufz"ahlbare Sprache und $W$ eine entscheidbare Sprache, so ist $L\ff W$ aufz"ahlbar.
\end{satz}
\begin{proof}
Es sei $U=\{u_i : 0 \le i \le n \wedge u_i \in L\}$ eine Aufz"ahlung von $L$.\\
Wir konstruieren einen Algorithmus, welcher die Eingabe $w$ akzeptiert genau dann, wenn $w\in L\ff W$.
Ist $w\notin W$ so kann nach Definition auch nicht $w\in L\ff W$ gelten. \\\\Wir w"ahlen das Wort $u_i$ aus der Aufz"ahlung $U$ von $L$ aus (\# 1).
Gilt nun $u_i\sqsubseteq w$, so wird eine Arbeitskopie $v$ mit $v:=w$ erstellt (\# 2). Im n"achsten Schritt pr"ufen wir f"ur alle Pr"afixe $v$ mit $u_i\sqsubseteq v\sqsubset w$ sukzessive, ob $v\in W$ (\# 3). 
\\\\Sollte dies der Fall sein, dann haben wir mit $v$ ein Pr"afix von $w$, welches gegen das $\min$ Kriterium verst"o"st. Dann verlassen wir die innere Schleife und wählen in der FOR-Schleife das n"achste $u_{i+1}$ aus(\# 4) und der Algorithmus l"auft \emph{ad infinitum}.% und der Algorithmus l"auft mit diesem $u_i$ weiter.
\\Gilt aber f"ur alle Pr"afixe $v\notin W$, so erreicht die innere Schleife ihre Abbruchbedingung und wir akzeptieren die Eingabe $w$ (\# 5).


\begin{algorithm}
\caption{akzeptiere $L\ff W$}
\label{split2}
\begin{algorithmic}
%\REQUIRE Menge X, int stufe
%Input: w
\STATE Input $w$
\STATE Sei
\IF{($w \notin W$)}  
\STATE $T$ rejects
\ENDIF\\
\COMMENT{\# 1}
\FOR{$i = 0$ to $\infty$} 
%\STATE z"ahle ein $u\in L$ auf 
\IF{  $u_i\in\pref(\{w\})$ }
\STATE $v := w$ \COMMENT{\# 2}
\WHILE{$v\neq u_i$}
\STATE $v \leftarrow cut(v)$ \COMMENT{\# 3}
\STATE \textbf{EXIT IF} $v\in W$ \COMMENT{\# 4}
\ENDWHILE
\IF{$v=u_i$}
\STATE $T$ accepts \COMMENT{\# 5}
\ENDIF
\ENDIF
\ENDFOR

\end{algorithmic}
\end{algorithm}
\end{proof}
\newpage
\subsection{L entscheidbar, W aufz"ahlbar}

\begin{satz}
Ist $L$ eine entscheidbare Sprache und $W$ eine aufz"ahlbare Sprache, so ist $L\ff W$ nicht notwendig aufz"ahlbar.
\end{satz}
\begin{proof}
Wir w"ahlen ein $A\subseteq \mathbb{N}$ welches aufz"ahlbar, aber nicht entscheidbar ist und
setzen $W$ wie in (\ref{last1}) und $L$ wie in (\ref{last2}). $W$ ist also aufz"ahlbar und $L$ sogar regul"ar und endlich.
Angenommen (\ref{last3}) w"are aufz"ahlbar, so w"are (\ref{last4}) auch aufz"ahlbar. Das w"urde bedeuten, dass $\mathbb{N}\setminus A$ aufz"ahlbar und damit $A$ entscheidbar w"are. Dies ergibt aber einen Widerspruch zur Annahme, da wir $A$ so gew"ahlt haben, dass es nicht entscheidbar ist.
\setcounter{equation}{0}
\begin{eqnarray}
W&=&\{0^{n+1}\ 1^{n+1} : n\in \mathbb{N} \} \cup \{0^{n+1}\ 1 : n\in A \} \label{last1} \\
L&=&\{0\} \label{last2} \\
L\ff W = \{0^{n+1}\ 1^{n+1} &:& n\in\mathbb{N} \wedge n\notin A\} \cup \{0^{n+1}\ 1:n\in \mathbb{N} \wedge n\in A\} \label{last3} \\
(L\ff W)\cap \{0^n\ 1^n : n\in \mathbb{N}\} &=& \{0^{n+1}\ 1^{n+1} : n\in \mathbb{N} \wedge n\notin A\} \label{last4} 
\end{eqnarray}
\end{proof}
