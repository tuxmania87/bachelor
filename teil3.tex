\subsection{Regularit�t}
Seien $L$ und $W$ regul\"ar, so ist auch $L\ff W$ regul\"ar.\\\\
Automat $A_L= (X,Z,z_{0},\delta_L,Z_f)$ akzeptiere $L$, Automat $A_W = (X,S,s_{0},f,S_{f})$ akzeptiere $W$.
Automat $A$ akzeptiert $L\ff W$, \\\\Vorgehensweise:\\
$A_L$ und $A_W$ lesen das Wort $w$ parallel. Falls $A_L$ akzeptiert und w�hlt $A$ nicht-deterministisch aus ob Schritt 2 aktiviert wird oder nicht.\\
Schritt 2: $A_W$ liest das Wort $w$ zu Ende, w\"ahrend $A_L$ im Zustand $z_f'$ verweilt. Sollte $A_W$ auf diesem mehr als einmal akzeptieren, so akzeptiert $A$ nicht indem $A_W$ im Stoppzustand $s_x$ stehen bleibt, ansonsten akzeptiert $A$.\\

$A = (X,Z\cup \{ z_f'\}\times S\cup \{ s_x\}, (z_{0},s_{0} ), \delta , \{ (z_f',s') : s'\in S_f \} ), s_x\notin S$
mit\\\\
$\delta = \{ ((z_i,s_i),x,(z_j,s_j)) : (z_i,x,z_j) \in \delta_L \wedge f(s_i,x)=s_j  \} \cup \\
 \{ ((z_i,s_i),x,(z_f',s_j)) : (z_i,x,z') \in \delta_L \wedge z'\in Z_f \wedge f(s_i,x)=s_j  \} \cup \\
 \{ ((z_f',s_i),x,(z_f',s_j)) : f(s_i,x)=s_j  \wedge s_i \notin S_f\} \cup  \\
 \{ ((z_f',s_i),x,(z_f',s_x)) : f(s_i,x)=s_j  \wedge s_i \in S_f\} 
 $\\
\begin{proof}
Der konstruierte Automat $A$ akzeptiert nur in einem Zustand $(z_f',s'), s'\in S_f$
Nach Konstruktion gelangt $A$ bei Eingabe $w$ genau dann in $(z_f',s)$, wenn ein Wort $l\in L$ mit $l\sqsubseteq w$ existiert.
In solch einem Fall kann der Automat umschalten. Wenn dies der Fall ist, so arbeitet $A$ weiter auf der Eingabe $w$ wie $A_W$ es tut.
Sollte $A_W$ nun akzeptieren und $w$ ist noch nicht zu Ende gelesen, so wird $A$ nach Konstruktion in einen Stoppzustand $(z_f',s_x)$ geleitet, in dem er nie wieder akzeptiert.
$A$ akzeptiert also nur wenn $A_L$ akzeptiert hat (es existiert ein $l\in L \wedge l\sqsubseteq w$) und wenn f�r alle $v'$ mit $l\sqsubseteq v' \sqsubset w$ gilt $v'\notin W$.\\\\
Demnach akzeptiert $A$ die Eingabe $w$ genau dann, wenn $w\in L\ff W$
\end{proof}

\subsection{Kontextfreiheit}

\subsubsection{deterministisch kontextfrei}

Es existieren deterministisch kontextfreie Sprachen $L,W$, sodass $L\ff W$ nicht deterministisch kontextfrei ist!\\\\
Sei $L=\{a^nb^nc^i:i,n>0\}$ und $W=\{a^ib^nc^n:i,n>0\}$. Sowohl $L$ als auch $W$ sind deterministische, kontextfreie und auch lineare Sprachen, da es je einen deterministischen Kellerautomaten gibt, der $L$ sowie $W$ akzeptiert und eine es lineare Grammatiken $G_L$ und $G_W$ gibt, sodass $L(G_L) = L$ und $L(G_W) = W$ gilt.\\
Betrachtet man sich nun ein Wort $u\in L\ff W$ so muss $u$ laut Definition folgende Struktur besitzen $u\in \min( l\cdot X^* \cap W)\textit{ f�r ein }l \in L$.
Damit sieht man leicht, dass $L\ff W= \{a^nb^nc^n:n>0\}$ und $\{a^nb^nc^n:n>0\}$ ist bekanntlich nicht kontextfrei, also auch nicht deterministisch kontextfrei

\subsection{Entscheidbarkeit}
\subsubsection{L und W entscheidbar}
Seien $L$ und $W$ (Turing)entscheidbar, so ist auch $L\ff W$ entscheidbar.\\\\
Seien die Turing Maschinen $T_L$ und $T_W$.\\
Die Turing Maschine $T$ entscheidet $L\ff W$ nach folgendem Algorithmus:\\

\begin{algorithm}
\caption{entscheide $L\ff W$, Input $w$}
\label{split}
\begin{algorithmic}
%\REQUIRE Menge X, int stufe
%Input: w
\IF{($w \notin W$)}  
\STATE $T$ rejects
\ELSE
\IF{($w \in L$)}
\STATE $T$ accepts
\ENDIF
\ENDIF
\STATE $w' = w$
\REPEAT
\STATE $w' \leftarrow \mathit{cut}(w')$
\IF{($w'\in W$)}
\STATE $T$ rejects
\ENDIF
\IF{($w' \in L$)}
\STATE $T$ accepts
\ENDIF
\UNTIL{($w'==e$)}
\STATE $T$ rejects
\end{algorithmic}
\end{algorithm}

\subsubsection{L akzeptierbar, W entscheidbar}

Sei $w$ die Eingabe. Es soll $w$ akzeptiert werden, wenn $w\in L\ff W$ ist, wobei $L$ akzeptierbar und $W$ entscheidbar ist.\\\\
Vorgehensweise:\\
Zun�chst wird gepr�ft ob $w\notin W$ liegt, falls das der Fall ist, so kann nach Definiton nicht akzeptiert werden.\\
Nun werden alle Pr�fixe von $w$ auf jeweils ein Band geschrieben, sodass man bei $\vert w\vert = n$, $n$-B�nder ben�tigt.
Dann wird in einer Endlosschleife jeweils auf jedem Band die Maschine die $L$-akzeptiert ausgef�hrt.\\\\

Notiz: $\vert w\vert = n$\\
1. wenn $w\notin W$, laufe in Endlosschleife\\
2. Kopiere alle $u$ mit $u\sqsubset w$ auf jeweils ein Band.\\
3. \\
\begin{verbatim}
while(true) do
  for(i=1 to n) do 
    if(T_W accepts on tape i) T rejects
    do 1 takt with $T_L$ on tape i, if T_W accepts then T accept
  endfor
endwhile
\end{verbatim}


\subsubsection{L entscheidbar, W akzeptierbar}

Sei $L$ entscheidbar und $W$ aufz�hlbar, so ist $L\ff W$ nicht notwendigerweise aufz�hlbar.\\
\begin{proof}
Bemerkung: $A$ ist aufz�hlbar, aber nicht entscheidbar\\\\
Sei $W=\{0^{n+1}\ 1^{n+1} : n\in \mathbb{N} \} \cup \{0^{n+1}\ 1 : n\in A \}$, also $W$ ist aufz�hlbar und sei $L=\{0\}$.\\
So ist $L\ff W = 0 \ff W = \{0^{n+1}\ 1^{n+1} : n\in\mathbb{N} \wedge n\notin A\} \cup \{0^{n+1}\ 1:n\in \mathbb{N} \wedge n\in A\}$. 
Angenommen $0\ff W$ w�re aufz�hlbar, so m�sste der Schnitt mit einer aufz�hlbaren Sprache wieder aufz�hlbar sein. 
Sei $L_A = \{0^n\ 1^n : n\in\mathbb{N}\}$ offensichtlich aufz�hlbar. Dann ergibt sich aber f�r $0\ff W \cap L_A = \{0^{n+1}\ 1^{n+1} : n\in\mathbb{N} \wedge n\notin A\}$, was nicht aufz�hlbar ist.
\end{proof}