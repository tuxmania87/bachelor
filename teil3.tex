\subsection{Regularit�t}
\begin{satz}
Sind $L$ und $W$ regul\"ar, so ist auch $L\ff W$ regul\"ar.
\end{satz}
Der nicht-deterministische Automat $A_L= (X,Z,z_{0},\delta_L,Z_f)$ akzeptiere $L$, der deterministische Automat $A_W = (X,S,s_{0},f,S_{f})$ akzeptiere $W$.
Wir konstruieren einen nicht-determninistischen Automaten $A$, der $L\ff W$ akzeptiert \\\\Arbeitsweise:\\
$A_L$ und $A_W$ lesen das Wort $w$ parallel. Falls $A_L$ ein Pr�fix $v$ von $w$ akzeptiert, w�hlt $A$ nicht-deterministisch aus ob Schritt 2 aktiviert wird oder nicht.\\
Wird Schritt zwei aktiviert, so liest $A_W$ das Wort $w$ weiter, w�hrend $A_L$ in einem Ruhezustand $z_f'$ verweilt. Akzeptiert $A_W$ nun ein Pr�fix $v',\ v\sqsubseteq v'\sqsubset w$, so lehnt $A$ das Eingabewort $w$ ab. Sollte der Automat $A_W$ kein solches Pr�fix $v'$ akzeptieren, aber das Eingabewort $w$, so akzeptiert auch $A$.\\

$A = (X,Z\cup \{ z_f'\}\times S\cup \{ s_x\}, (z_{0},s_{0} ), \delta , \{ (z_f',s') : s'\in S_f \} ), s_x\notin S$
mit\\\\
$\delta = \{ ((z_i,s_i),x,(z_j,s_j)) : (z_i,x,z_j) \in \delta_L \wedge f(s_i,x)=s_j  \} \cup \\
 \{ ((z_i,s_i),x,(z_f',s_j)) : (z_i,x,z') \in \delta_L \wedge z'\in Z_f \wedge f(s_i,x)=s_j  \} \cup \\
 \{ ((z_f',s_i),x,(z_f',s_j)) : f(s_i,x)=s_j  \wedge s_i \notin S_f\} \cup  \\
 \{ ((z_f',s_i),x,(z_f',s_x)) : f(s_i,x)=s_j  \wedge s_i \in S_f\} 
 $\\
\begin{proof}
Nach Konstruktion ist klar, dass der Automat $A$ nur W�rter aus $L\ff W$ akzeptiert. Er akzeptiert auch alle W�rter aus $L\ff W$, weil der Automat $A_L$, welcher die Pr�fixe akzeptiert, nicht-deterministisch entscheidet, ob das gefunden Pr�fix zum Wort $w$ geh�rt.
Damit akzeptiert der Automat $A$ ein Eingabewort $w$ genau dann, wenn $w\in L\ff W$
\end{proof}
\newpage
\subsection{Kontextfreiheit}

\subsubsection{deterministisch kontextfrei}

Es existieren deterministisch kontextfreie Sprachen $L,W$, sodass $L\ff W$ nicht deterministisch kontextfrei ist.\\\\
Sei $L=\{a^nb^nc^i:i,n>0\}$ und $W=\{a^ib^nc^n:i,n>0\}$. Sowohl $L$ als auch $W$ sind deterministische, kontextfreie und auch lineare Sprachen, da es je einen deterministischen Kellerautomaten gibt, der $L$ sowie $W$ akzeptiert und eine es lineare Grammatiken $G_L$ und $G_W$ gibt, sodass $L(G_L) = L$ und $L(G_W) = W$ gilt.\\
Betrachtet man sich nun ein Wort $u\in L\ff W$ so muss $u$ laut Definition folgende Struktur besitzen $u\in \min( v\cdot X^* \cap W)\textit{ f�r ein }v \in L$.
Damit sieht man leicht, dass $L\ff W= \{a^nb^nc^n:n>0\}$ und diese Sprache ist bekanntlich nicht kontextfrei, also auch nicht deterministisch kontextfrei
\newpage
\subsection{Entscheidbarkeit}
\subsubsection{L und W entscheidbar}
Seien $L$ und $W$ (Turing)entscheidbar, so ist auch $L\ff W$ entscheidbar.\\\\
Seien die Turing Maschinen $T_L$ und $T_W$.\\
Die Turing Maschine $T$ entscheidet $L\ff W$ nach folgendem Algorithmus:\\

\begin{algorithm}
\caption{entscheide $L\ff W$, Input $w$}
\label{split}
\begin{algorithmic}
%\REQUIRE Menge X, int stufe
%Input: w
\IF{($w \notin W$)}  
\STATE $T$ rejects
\ELSE
\IF{($w \in L$)}
\STATE $T$ accepts
\ENDIF
\ENDIF
\STATE $w' = w$
\REPEAT
\STATE $w' \leftarrow \mathit{cut}(w')$
\IF{($w'\in W$)}
\STATE $T$ rejects
\ENDIF
\IF{($w' \in L$)}
\STATE $T$ accepts
\ENDIF
\UNTIL{($w'==e$)}
\STATE $T$ rejects
\end{algorithmic}
\end{algorithm}
\newpage
\subsubsection{L akzeptierbar, W entscheidbar}

Sei $w$ die Eingabe. Es soll $w$ akzeptiert werden, wenn $w\in L\ff W$, wobei $L$ akzeptierbar und nicht entscheidbar und $W$ entscheidbar.\\
Dazu z�hlt man $l\in L$ auf und pr�ft f�r alle $u$ mit $l\sqsubseteq u\sqsubset w$, ob ein $u \in W$ liegt. Sollte dies der Fall sein, darf man nicht akzeptieren. Dann wird ein weiteres $l$ aufgez�hlt und der Algorithmus beginnt von vorn.\\

\begin{algorithm}
\caption{akzeptiere $L\ff W$, Input $w$}
\label{split2}
\begin{algorithmic}
%\REQUIRE Menge X, int stufe
%Input: w
\IF{($w \notin W$)}  
\STATE $T$ rejects
\ENDIF
\WHILE{true} 
\STATE z�hle ein $u\in L$ auf
\IF{  $u\in\pref(\{w\})$ }
\STATE $v := w$
\WHILE{$v\neq u$}
\STATE $v := w'$ mit $w'\cdot x = v,\, x\in X$
\STATE exit WHEN $v\in W$
\ENDWHILE
\IF{$v=u$}
\STATE $T$ accepts
\ENDIF
\ENDIF
\ENDWHILE

\end{algorithmic}
\end{algorithm}

\subsubsection{L entscheidbar, W akzeptierbar}

Sei $L$ entscheidbar und $W$ aufz�hlbar, so ist $L\ff W$ nicht notwendig aufz�hlbar.\\
\begin{proof}
Wir w�hlen ein $A\subseteq \mathbb{N}$ welches aufz�hlbar, aber nicht entscheidbar ist und
setzen $W=\{0^{n+1}\ 1^{n+1} : n\in \mathbb{N} \} \cup \{0^{n+1}\ 1 : n\in A \}$. Also ist $W$ aufz�hlbar.
Betrachten wir nun $L\ff W = 0 \ff W = \{0^{n+1}\ 1^{n+1} : n\in\mathbb{N} \wedge n\notin A\} \cup \{0^{n+1}\ 1:n\in \mathbb{N} \wedge n\in A\}$. 
Angenommen $0\ff W$ w�re aufz�hlbar, so m�sste $0\ff W\cap V = \{0^{n+1}\ 1^{n+1} : n\in \mathbb{N} \wedge n\notin A\}$ wiederum aufz�hlbar sein. Weil aber laut Vorraussetzung $A$ nicht entscheidbar ist, kann $X^*\backslash A$ nicht aufz�hlbar sein. Widerspruch zur Annahme.
\end{proof}