\subsection{Allgemeines}

In diesem Abschnitt betrachten wir Eigenschaften der Operation der Fortsetzung von Sprachen.
Hierbei werden insbesondere die Stabilit�t bzw. Monotonie bez�glich mengentheoretischen Operationen betrachtet.

\subsection{grundlegende Beziehungen}

Aus der Definition folgen direkt die Eigenschaften:

\begin{eigen}\label{eig1}
$\{w\} \ff L = \{w\}$, wenn $w\in L$\\
$W\cap L \subseteq W\ff L \subseteq L$
\end{eigen}

\begin{eigen}\label{cupeig}
$(W\cup V)\ff L = W\ff L\cup V\ff L$
\end{eigen}

Aus Eigenschaft \ref{eig1} wissen wir, dass $W\cap L \subseteq W\ff L$. Man kann nun die Fortsetzung von $W$ in $L$ aufsplitten, sodass sich folgende Beziehung ergibt:

\begin{eigen}\label{basiseigen}
$W\ff L = W\cap L \cup (W\backslash L) \ff L$
\end{eigen}

Aus der Eigenschaft \ref{basiseigen} folgt durch Einschr�nkung der Mengen $W$ bzw. $L$:

\begin{folg}
$W\subseteq L \to W\ff L = W\cap L = W$
\end{folg}
\begin{folg}
$W\supseteq L \to W\ff L = W\cap L = L$
\end{folg}


Aus Eigenschaft \ref{cupeig} folgt:
\begin{eigen}
$W\ff L \to W\ff V \subseteq L \ff V$
\end{eigen}

Damit haben wir eine
\begin{gleich}
$(L\cap U) \ff V \subseteq (L\ff V) \cap (U\ff V)$
\end{gleich}
Dabei muss nicht notwendig die Gleichheit gelten, wie das folgende Beispiel zeigt
\begin{beispiel}
Es seien $L= \{aa,bb\}\, ,U = \{aa,b\}\, ,V = \{aa,bb\}$. \\Dann haben wir $(L\cap U) \ff V = \{aa\} \subset \{aa,bb\} = (L\ff V) \cap (U\ff V)$
\end{beispiel}

%\begin{proof}
%$$w\in  (L\cap U)\ff V \to w\in V \wedge \exists p: p\sqsubseteq w \wedge p\in L \wedge p\in U \wedge p\textit{ ist minimal}$$
%$$\to \underbrace{w\in V \wedge \exists p: p\sqsubseteq w \wedge p\in L \wedge p\textit{ ist minimal}}_{w\in L\ff V} 
%\wedge \underbrace{w\in V \wedge \exists p: p\sqsubseteq w \wedge p\in U \wedge p\textit{ ist minimal}}_{w\in U\ff V} 
%$$
%\end{proof}


\begin{gleich}
$L\ff (U\cup V)\subseteq (L\ff U) \cup (L\ff V)$
\end{gleich}
Zum Beweis bem�hen wir die Beziehung aus 
\begin{lem}
$\min (A\cup B) \subseteq \min A \cup \min B$
\end{lem}
\begin{proof}
Es gen�gt, die Eigenschaft f�r $L=\{w\}$ zu zeigen.\\
Es gilt $w \ff (W\cup V)=\min (w\cdot X^* \cap (W\cup V)) =\min ((w\cdot X^* \cap W) \cup (w\cdot X^* \cap V))$\\
und mit Lemma 1 erhalten wir:\\
$ w\ff (W\cup V) \subseteq \min (w\cdot X^* \cap W) \cup \min (w\cdot X^* \cap V)=(L\ff W)\cup (L\ff V)$
\end{proof}
Dabei muss nicht notwendig die Gleichheit gelten, wie das folgende Beispiel zeigt.
\begin{beispiel}
Es seien $L= \{a,b\}\, ,U =\{aaa\}\, ,V = \{bb,aa\}$. \\Dann haben wir $L\ff (U\cup V)=\{aa,bb\} \subset \{aa,bb,aaa\} = (L\ff U) \cup (L\ff V)$
\end{beispiel}

Eine �hnliche Beziehung ergibt sich f�r die Operation $\cup$

\begin{gleich}
$L\ff (U\cap V)\supseteq (L\ff U) \cap (L\ff V)$
\end{gleich}
Zum Beweis bem�hen wir die Beziehung aus 
\begin{lem}
$\min(A\cap B) \supseteq \min A \cap \min B$
\end{lem}
%\begin{lem}
%$$L\ff U \cap L \ff V \stackrel{!}{=} \bigcup_{l\in L} (l\ff U \cap l\ff V)$$
%\end{lem}
\begin{proof}
Es gen�gt die Eigenschaft f�r $L=\{w\}$ zu zeigen.\\
Es gilt $w\ff (U\cap V) = \min(w\cdot X^* \cap (U \cap V))= \min((w\cdot X^* \cap U) \cap (w\cdot X^* \cap V))$\\
und mit Lemma 2 erhalten wir\\
$ w\ff (U\cap V) \supseteq \min(w\cdot X^* \cap U) \cap \min(w\cdot X^* \cap V)$\\
$=(L\ff U) \cap (L\ff V)$
\end{proof}
Dabei muss nicht notwendig die Gleichheit gelten, wie das folgende Beispiel zeigt
\begin{beispiel}
Es seien $L= \{a,b\}\, ,U =\{aaa,b,bb\}\, ,V = \{bb,aaa\}$\\
Dann haben wir $L\ff (U\cap V) = \{bb,aaa\} \supset \{ aaa \} =(L\ff U) \cap (L\ff V)$
\end{beispiel}

\newpage
\subsection{Konkatenation}
Bisher wurden haupts�chlich die Operationen $\cap$ sowie $\cup$ in Verbindung mit $\ff$ betrachtet.
F�r die Konkatenation ergeben sich keine Eigenschaften allgemeing�ltiger Natur.\\

So gibt es Sprachen, bei denen die Gleichheit in folgender Weise gegeben ist:
\begin{beispiel}
Es seien  $L=\{e\}\quad ,U=\{a\}\quad ,V=\{b\} $, so erhalten wir \\
$L\ff (U\cdot V) \{ ab\} =  \{ab\} = (L\ff U) \cdot (L\ff V)$, also $L\ff (U\cdot V) = (L\ff U) \cdot (L\ff V)$
\end{beispiel}
weiterhin k�nnen wir Sprachen angeben, sodass Teilmengenbeziehung in folgender Weise existieren:
\begin{beispiel}
Es seien $L=\{a\}\quad ,U=\{aa\}\quad ,V=\{b,a\} $\\
Dann erhalten wir $L\ff (U\cdot V) = \{aab,aaa\} \supset  \{aaa\} = (L\ff U) \cdot (L\ff V)$, also $L\ff (U\cdot V) \supset (L\ff U) \cdot (L\ff V)$.
\end{beispiel}

Das letzte Beispiel in diesem Abschnitt zeigt deutlich, dass keine allgemeing�ltigen Eigenschaften f�r die Operation $\cdot$ bez�glich $\ff$ existieren:
\begin{beispiel}
Es seien $L=\{aab,a\}\quad U=\{aa\}\quad V=\{b\}$\\
Dann erhalten wir $L\ff (U\cdot V) = \{aab\} \neq \{aa\} = (L\ff U) \cdot (L\ff V)$, das bedeutet also, dass beide Seiten mengentheoretisch unvergleichbar sind, also 
$L\ff (U\cdot V) \not\supset (L\ff U) \cdot (L\ff V)$, sowie $L\ff (U\cdot V) \not\subset (L\ff U) \cdot (L\ff V)$
\end{beispiel}
Betrachtet man nun die Konkatenation `vorn`, also die Beziehung $(L\cdot U)\ff V$ zu  $(L\ff V) \cdot (U\ff V)$, so sieht man leicht, dass es sich auf 
der linken Seite der Gleichung um W�rter aus $V$ handelt die man vergleicht mit W�rtern aus $V^2$. \\Demnach treten hier Eigenschaften nur auf wenn $V$ eine ganz spezielle Gestalt hat.
\subsection{Gilt nicht...}
Folgt direkt aus den Gleichungen in 2.1
$$L\ff (U\cup V)\supset (L\ff U) \cup (L\ff V)$$
$$L\ff (U\cap V)\subset (L\ff U) \cap (L\ff V)$$
$$L\ff (U\cdot V)\subset (L\ff U) \cdot (L\ff V)$$
$$(L\cap U) \ff V \supset (L\ff V) \cap (U\ff V)$$

%\newpage
%$$L\ff (U\cup V)\subset (L\ff U) \cup (L\ff V)$$
%$$L= \{a,b\} U =\{aaa\} V = \{bb,aa\}$$\\
%$$L\ff (U\cup V) = (L\ff U) \cup (L\ff V)$$
%$$L= \{a,b\} U =\{aaa\} V = \{aaa\}$$\\
%$$L\ff (U\cap V)\supset (L\ff U) \cap (L\ff V)$$
%$$L= \{a,b\} U =\{aaa,b,bb\} V = \{bb,aaa\}$$\\
%$$L\ff (U\cap V)= (L\ff U) \cap (L\ff V)$$
%$$L= \{a\} U =\{aaa\} V = \{aaa\}$$\\
%$$(L\cap U) \ff V = (L\ff V) \cap (U\ff V)$$
%$$L= \{aaa\} U =\{aaa\} V = \{aaa\}$$\\
%\vspace{5mm}
%$$L\ff (U\cup V) \not\supset (L\ff U) \cup (L\ff V),$$
%$$\text{Gegenbeispiel: } L=\{a\}\quad U=\{abb,aaba\}\quad V=\{aab,aba\} $$
%$$L\ff (U\cup V) = \{abb,aba,aab\}\text{ ,aber } L\ff U \cup L\ff V = \{abb,aaba\} \cup \{aab,aba\} = \{abb,aaba,aab,aba\} $$
%\vspace{5mm}
%$$L\ff (U\cap V) \not\subset (L\ff U) \cap (L\ff V),$$
%$$\text{Gegenbeispiel: } L=\{a,b\}\quad U=\{a,aa\}\quad V=\{aa,b\} $$
%$$L\ff (U\cap V) = \{aa\}\text{ ,aber } L\ff U \cap L\ff V =
%\{a\} \cap \{b\} = \emptyset $$
%\vspace{5mm}
%$$L\ff (U\cdot V)= (L\ff U) \cdot (L\ff V),$$
%$$\text{Beispiel: } L=\{e\}\quad U=\{a\}\quad V=\{b\} $$
%$$L\ff (U\cdot V) \supset (L\ff U) \cdot (L\ff V),$$
%$$\text{Beispiel: } L=\{a\}\quad U=\{aa\}\quad V=\{b,a\} $$
%\vspace{5mm}
%$$(L_1 \ff L_2) \ff L_3 \nsupseteq L_1 \ff (L_2 \ff L_3)$$
%$$\text{Gegenbeispiel: } L_1=\{ab,aa\}\quad L_2=\{a,ab\}\quad L_3=\{aa\}$$
%$$(L_1 \ff L_2) \ff L_3 = \{ab\} \ff \{aa\} = \emptyset\text{ ,aber } 
%\{ab,aa\} \ff \{aa\} = \{aa\} $$
