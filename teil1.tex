\subsection{Gilt f�r alle Sprachen}

Folgende Eigenschaft folgt direkt aus der Definition:

\begin{gleich}
$$u\in L \to (u\ff L=\{u\})$$
\end{gleich}

Aus 2.1.1 folgt direkt
\begin{gleich}
$$U\ff L \subseteq L$$
\end{gleich}

Eine unmittelbare Folgerung aus der Definition 5 ergibt sich:
\begin{gleich}
$$(L\cup U)\ff V = (L\ff V) \cup (U\ff V)$$
\end{gleich}
%\begin{proof}
%$$(L\cup U) \ff V = \bigcup_{l\in L\cup U} \min(l\cdot X^* \cap V)$$
%$$ =\bigcup_{l\in L} \min(l\cdot X^* \cap V) \cup \bigcup_{l\in U} \min(l\cdot X^* \cap V)$$
%$$ = (L\ff V) \cup (U\ff V)$$
%\end{proof}

Aus dieser Gleichung 2.1.3 folgt wiederum direkt:

\begin{gleich}
$$L_2\subseteq L_1 \to L_2\ff W \subseteq L_1 \ff W$$
\end{gleich}
%\begin{proof}
%$$L_1\ff W = \bigcup_{l\in L_1} l\ff W$$
%$$=\bigcup_{l\in L_2} l\ff W \cup \bigcup_{l\in L_1\backslash L_2} l\ff W$$
%$$\supseteq \bigcup_{l\in L_2} l\ff W = L_2\ff W$$
%\end{proof}

Auf Gleichung 2.1.4 folgt direkt:

\begin{gleich}
$$(L\cap U) \ff V \subseteq (L\ff V) \cap (U\ff V)$$
\end{gleich}
Dabei muss nicht notwendigerweise die Gleichheit, wie das folgende Beispiel zeigt
$$\textit{Sei }L= \{aa,bb\}\, U = \{aa,b\}\, V = \{aa,bb\}$$
$$(L\cap U) \ff V = \{aa\} \subset \{aa,bb\} = (L\ff V) \cap (U\ff V)$$

%\begin{proof}
%$$w\in  (L\cap U)\ff V \to w\in V \wedge \exists p: p\sqsubseteq w \wedge p\in L \wedge p\in U \wedge p\textit{ ist minimal}$$
%$$\to \underbrace{w\in V \wedge \exists p: p\sqsubseteq w \wedge p\in L \wedge p\textit{ ist minimal}}_{w\in L\ff V} 
%\wedge \underbrace{w\in V \wedge \exists p: p\sqsubseteq w \wedge p\in U \wedge p\textit{ ist minimal}}_{w\in U\ff V} 
%$$
%\end{proof}


\begin{gleich}
$$L\ff (U\cup V)\subseteq (L\ff U) \cup (L\ff V)$$
\end{gleich}
\begin{lem}
$$\min (A\cup B) \subseteq \min A \cup \min B$$
\end{lem}
\begin{proof}
Es gen�gt, die Eigenschaft f�r $L=\{w\}$ zu zeigen.
$$w \ff (W\cup V)=\min (w\cdot X^* \cap (W\cup V))$$
Nach Anwenden der Distributivgesetze ergibt sich:
$$=\min ((w\cdot X^* \cap W) \cup (w\cdot X^* \cap V))$$
und mit Lemma 1 erhalten wir:
$$\subseteq \min (w\cdot X^* \cap W) \cup \min (w\cdot X^* \cap V)$$
$$=(L\ff W)\cup (L\ff V)$$
\end{proof}
Dabei muss nicht notwendigerweise die Gleichheit gelten, wie das folgende Beispiel zeigt.
$$\textit{Sei }L= \{a,b\}\, U =\{aaa\}\, V = \{bb,aa\}$$
$$L\ff (U\cup V)=\{aa,bb\} \subset \{aa,bb,aaa\} = (L\ff U) \cup (L\ff V)$$


\begin{gleich}
$$L\ff (U\cap V)\supseteq (L\ff U) \cap (L\ff V)$$
\end{gleich}
\begin{lem}
$$\min(A\cap B) \supseteq \min A \cap \min B$$
\end{lem}
%\begin{lem}
%$$L\ff U \cap L \ff V \stackrel{!}{=} \bigcup_{l\in L} (l\ff U \cap l\ff V)$$
%\end{lem}
\begin{proof}
Es gen�gt die Eigenschaft f�r $L=\{w\}$ zu zeigen.
$$w\ff (U\cap V) = \min(w\cdot X^* \cap (U \cap V))$$
Nach Anwenden der Distributivgesetze ergibt sich
$$= \min((w\cdot X^* \cap U) \cap (w\cdot X^* \cap V))$$
und mit Lemma 2 erhalten wir
$$\supseteq \min(w\cdot X^* \cap U) \cap \min(w\cdot X^* \cap V)$$
$$=(L\ff U) \cap (L\ff V)$$
\end{proof}
Dabei muss nicht notwendigerweise die Gleichheit gelten, wie das folgende Beispiel zeigt
$$\textit{Sei }L= \{a,b\}\, U =\{aaa,b,bb\}\, V = \{bb,aaa\}$$
$$L\ff (U\cap V) = \{bb,aaa\} \supset \{ aaa \} =(L\ff U) \cap (L\ff V)$$


\begin{gleich}
$$L_1\subseteq L_2 \to L_1\ff L_2 = L_1$$
\end{gleich}
\begin{eigen}
$$l\in L \to l\ff L = \{l\}$$
\end{eigen}
\begin{proof}
$$L_1\ff L_2 = \bigcup_{l\in L_1} l\ff L_2$$
$$\textit{wegen } L_1\subseteq L_2: l\ff L_2 = \{l\}$$
$$ = \bigcup_{l\in L_1} \{l\} = L_1$$
\end{proof}

\begin{gleich}
$$L_2\subseteq L_1 \to L_1\ff L_2 = L_2$$
\end{gleich}
\begin{proof}
$$\textit{Da }L_2\subseteq L_1\textit{ gilt f�r alle }l\in L_2 \wedge l\in L_1: l\ff L_2=\{l\}$$
$$\bigcup_{l\in L_1} l\ff L_2 = \bigcup_{l\in L_1} \{l\} = L_2$$
\end{proof}

\subsection{Konkatenation}
F�r die Konkatenation ergeben sich keine allgemeinen Eigenschaften.\\
So gibt es Sprachen derart, dass gilt
$$L\ff (U\cdot V) = (L\ff U) \cdot (L\ff V)$$
Beispiel:
$$\textit{Sei } L=\{e\}\quad U=\{a\}\quad V=\{b\} $$
$$L\ff (U\cdot V) \{ ab\} =  \{ab\} = (L\ff U) \cdot (L\ff V)$$\\
au�erdem gibt es andere Sprachen, sodass gilt
$$L\ff (U\cdot V) \supset (L\ff U) \cdot (L\ff V)$$
Beispiel:
$$\textit{Sei } L=\{a\}\quad U=\{aa\}\quad V=\{b,a\} $$
$$L\ff (U\cdot V) = \{aab,aaa\} \supset  \{aaa\} = (L\ff U) \cdot (L\ff V)$$\\
desweiteren gibt es Sprachen, sodass gilt
$$L\ff (U\cdot V) \neq (L\ff U) \cdot (L\ff V)$$
Beispiel:
$$\textit{Sei } L=\{aab,a\}\quad U=\{aa\}\quad V=\{b\}$$
$$L\ff (U\cdot V) = \{aab\} \neq \{aa\} = (L\ff U) \cdot (L\ff V)$$\\
Betrachtet man nun die Konkatenation `vorn`, also $(L\cdot U)\ff V ? (L\ff V) \cdot (U\ff V)$ so sieht man leicht, dass es sich auf 
der linken Seite der Gleichung um W�rter aus $V$ handelt die man vergleicht mit W�rtern aus $V^2$. Demnach treten hier Eigenschaften nur auf wenn $V$ eine ganz spezielle Gestalt hat.
\subsection{Gilt nicht...}
Folgt direkt aus den Gleichungen in 2.1
$$L\ff (U\cup V)\supset (L\ff U) \cup (L\ff V)$$
$$L\ff (U\cap V)\subset (L\ff U) \cap (L\ff V)$$
$$L\ff (U\cdot V)\subset (L\ff U) \cdot (L\ff V)$$
$$(L\cap U) \ff V \supset (L\ff V) \cap (U\ff V)$$

%\newpage
%$$L\ff (U\cup V)\subset (L\ff U) \cup (L\ff V)$$
%$$L= \{a,b\} U =\{aaa\} V = \{bb,aa\}$$\\
%$$L\ff (U\cup V) = (L\ff U) \cup (L\ff V)$$
%$$L= \{a,b\} U =\{aaa\} V = \{aaa\}$$\\
%$$L\ff (U\cap V)\supset (L\ff U) \cap (L\ff V)$$
%$$L= \{a,b\} U =\{aaa,b,bb\} V = \{bb,aaa\}$$\\
%$$L\ff (U\cap V)= (L\ff U) \cap (L\ff V)$$
%$$L= \{a\} U =\{aaa\} V = \{aaa\}$$\\
%$$(L\cap U) \ff V = (L\ff V) \cap (U\ff V)$$
%$$L= \{aaa\} U =\{aaa\} V = \{aaa\}$$\\
%\vspace{5mm}
%$$L\ff (U\cup V) \not\supset (L\ff U) \cup (L\ff V),$$
%$$\text{Gegenbeispiel: } L=\{a\}\quad U=\{abb,aaba\}\quad V=\{aab,aba\} $$
%$$L\ff (U\cup V) = \{abb,aba,aab\}\text{ ,aber } L\ff U \cup L\ff V = \{abb,aaba\} \cup \{aab,aba\} = \{abb,aaba,aab,aba\} $$
%\vspace{5mm}
%$$L\ff (U\cap V) \not\subset (L\ff U) \cap (L\ff V),$$
%$$\text{Gegenbeispiel: } L=\{a,b\}\quad U=\{a,aa\}\quad V=\{aa,b\} $$
%$$L\ff (U\cap V) = \{aa\}\text{ ,aber } L\ff U \cap L\ff V =
%\{a\} \cap \{b\} = \emptyset $$
%\vspace{5mm}
%$$L\ff (U\cdot V)= (L\ff U) \cdot (L\ff V),$$
%$$\text{Beispiel: } L=\{e\}\quad U=\{a\}\quad V=\{b\} $$
%$$L\ff (U\cdot V) \supset (L\ff U) \cdot (L\ff V),$$
%$$\text{Beispiel: } L=\{a\}\quad U=\{aa\}\quad V=\{b,a\} $$
%\vspace{5mm}
%$$(L_1 \ff L_2) \ff L_3 \nsupseteq L_1 \ff (L_2 \ff L_3)$$
%$$\text{Gegenbeispiel: } L_1=\{ab,aa\}\quad L_2=\{a,ab\}\quad L_3=\{aa\}$$
%$$(L_1 \ff L_2) \ff L_3 = \{ab\} \ff \{aa\} = \emptyset\text{ ,aber } 
%\{ab,aa\} \ff \{aa\} = \{aa\} $$
