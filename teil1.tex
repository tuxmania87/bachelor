In diesem Abschnitt betrachten wir Eigenschaften der Operation Fortsetzung f"ur formale Sprachen.
Hierbei wird zun"achst die algebraische Struktur $({2^X}^*,\ff)$ untersucht. Anschliessend untersuchen wir die Monotonie der Operation Fortsetzung bez"uglich $\subseteq$ und die Stabilit"at bez"uglich der mengentheoretischen Operationen $\cap$ und $\cup$.

\section{Eigenschaften der algebraischen Struktur $({2^X}^*,\ff)$}

\subsection{Kommutativit"at und Assoziativit"at}
Die algebraische Struktur $A=({2^X}^*,\ff)$ ist weder kommutativ, noch assoziativ. Um dies zu zeigen, wird ein Gegenbeispiel angegeben.

\vspace{2ex}

\begin{beispiel}
Es seien $L=\{aa,bb\},\ V = \{aa,b\}$ und $W=\{aa,bb\}$. \\Dann haben wir $(L\ff V)\ff W = \{aa\} \ff W = \{aa\}$, aber $L\ff (V\ff W) = L\ff \{aa,bb\} = \{aa,bb\}$ und $L\ff V = \{aa\}$, aber $V\ff L=\{aa,bb\}$.
\end{beispiel}

\subsection{Einselement}

Da die algebraische Struktur $A$ nicht kommutativ ist, werden bei der Untersuchung auf Einselemente sowohl linksneutrale, als auch rechtsneutrale Elemente betrachtet.
Wir beginnen mit linksneutralen Elementen.
Man erkennt leicht, dass $X^*$ ein solches Element ist, denn es gilt $X^*\ff L = L$.
Analog ist $X^*$ rechtsneutrales Element, denn es gilt $L\ff X^* = L$.
Nehmen wir nun an, es gebe weitere links- oder rechtsneutrale Elemente $E_l$ bzw. $E_r$, so haben wir $E_l=E_l\ff X^* = X^*$, da $X^*$ rechtsneutrales Element ist, sowie $E_r = X^*\ff E_r = X^*$, da $X^*$ auch linksneutrales Element ist.
\subsection{Nullelement}
Wir betrachten, da $A$ nicht kommutativ ist, sowohl linksabsorbierende, als auch rechtsabsorbierende Elemente.
Wir beginnen mit linksabsorbierenden Elementen.
Man erkennt leicht, dass $\emptyset$ ein solches Element ist, denn es gilt $\emptyset\ff L = \emptyset$.
Analog ist $\emptyset$ rechtsabsorbierendes Element, denn es gilt $L\ff\emptyset =\emptyset$.
Nehmen wir an, es gebe weitere links- oder rechtsabsorbierende Elemente $N_l$ bzw. $N_r$, so haben wir $N_l = N_l\ff \emptyset = \emptyset$, da $\emptyset$ rechtsabsorbierendes Element ist, sowie $N_r = \emptyset\ff N_r = \emptyset$, da $\emptyset$ auch linksabsorbierendes Element ist.\\\\
Damit kann die algebraische Struktur $A=({2^X}^*,\ff)$ aufgrund fehlender Assoziativit"at weder eine Gruppe noch eine Halbgruppe sein, besitzt aber ein Null- und ein Einselement.

\section{\emph{BOOLE}sche Operationen}
In diesem Abschnitt untersuchen wir die Operation Fortsetzung auf Stabilit"at bez"uglich den mengentheoretischen Operationen $\cap$ und $\cup$ und auf Monotonie bez"uglich $\subseteq$.
\\\\Aus der Definition folgt direkt die Eigenschaft

\vspace{2ex}

\begin{eigen}\label{eig1}
Es gilt genau dann $\{w\} \ff W = \{w\}$, wenn $w\in W$.
\end{eigen}

Aus Eigenschaft \ref{eig1} erhalten wir

\vspace{2ex}

\begin{eigen}
$(L\cap W) \ff W = L\cap W$
\end{eigen}
\begin{proof}
Wir wissen aus der Definition, dass $(L\cap W)\ff W = \bigcup_{w\in L\cap W} w\ff W$. Mit Eigenschaft \ref{eig1} erhalten wir dann $(L\cap W)\ff W = L\ff W$.
\end{proof}

Da die Operation Fortsetzung einer Sprache $L$ in eine Sprache $W$ "uber die Vereinigung $\cup$ aller W"orter aus $L$ definiert wurde, folgt aus der Definition

\vspace{2ex}

\begin{eigen}\label{cupeig}
%$(L\cup V)\ff W = L\ff W\cup V\ff W$
$(\bigcup_{i\in\mathbb{N}} L_i) \ff W = \bigcup_{i\in\mathbb{N}} (L_i\ff W)$
\end{eigen}
\begin{proof} 
Nach Definition gilt $(\bigcup_{i\in\mathbb{N}} L_i) \ff W = \bigcup_{w\in L_0\cup L_1\cup ... \cup L_n} w\ff W$\\
Wir betrachten nun die Vereinigung f"ur jede Sprache, so dass $\bigcup_{w\in L_0\cup L_1\cup ... \cup L_n} w\ff W = (\bigcup_{w\in L_0} w\ff W) \cup (\bigcup_{w\in L_1} w\ff W) \cup ... \cup (\bigcup_{w\in L_n} w\ff W)$. Damit haben wir $ (\bigcup_{i\in\mathbb{N}} L_i) \ff W  = \bigcup_{i\in\mathbb{N}} (L_i\ff W)$
\end{proof}

Wir zerlegen nun die Fortsetzung von $L$ in $W$ so, dass sich folgende Beziehung ergibt:

\vspace{2ex}

\begin{eigen}\label{basiseigen}
$L\ff W = (L\cap W) \cup ((L\setminus W) \ff W)$
\end{eigen}
\begin{proof}
Wir zerlegen zun"achst $L$ in $L\cap W$ und $L\setminus W$. Damit haben wir (\ref{basis1}). 
Nach Anwenden der Eigenschaft \ref{cupeig} erhalten wir (\ref{basis2}). Mit Eigenschaft \ref{eig1} l"asst sich der Ausdruck zu (\ref{basis3}) vereinfachen.
\begin{eqnarray}
L\ff W = ((L\cap W) \cup (L\setminus W)) \ff W \label{basis1}\\
L\ff W = ((L\cap W) \ff W) \cup ((L\setminus W) \ff W) \label{basis2} \\
L\ff W = (L\cap W) \cup ((L\setminus W) \ff W) \label{basis3} 
\end{eqnarray}
\end{proof}

Daraus folgt direkt

\vspace{2ex}

\begin{folg}\label{klaro}
$L\cap W \subseteq L\ff W \subseteq W$
\end{folg}


Wir bilden nun Inklusionsbeziehungen zwischen $L$ und $W$ und erhalten mit Eigenschaft \ref{basiseigen}, zwei Folgerungen.
Zu beachten ist, dass in Folgerung \ref{folg1} eine "`genau dann wenn"' Beziehung gilt.

\vspace{2ex}

\begin{folg}\label{folg1}
$L\subseteq W \leftrightarrow L\ff W = L\cap W = L$
\end{folg}

\vspace{2ex}

\begin{folg}\label{folg2}
$L\supseteq W \to L\ff W = W$
\end{folg}
F"ur Folgerung \ref{folg2} gilt nicht die umgekehrte Richtung. Dazu geben wir ein Beispiel an.

\vspace{2ex}

\begin{beispiel}
Es seien $L=\{b\}, W=\{bb\}$. Dann haben wir $L\ff W = W$, aber $L\not\supseteq W$.
\end{beispiel}

Da die Operation Fortsetzung nicht kommutativ ist, betrachten wir nun die Monotonie sowie die Stabilit"at der Operation bez"uglich Vorder- und Hinterglied getrennt.
\subsection{Eigenschaften im Vorderglied}

Wir betrachten zun"achst die Stabilit"at der Operation Fortsetzung bez"ulich $\cap$ und $\cup$ und die Monotonie bez"uglich $\subseteq$ im Vorderglied.\\\\
Aus Eigenschaft \ref{cupeig} folgt

\vspace{2ex}

\begin{eigen}\label{fresh}
$L\subseteq W \to L\ff V \subseteq W \ff V$
\end{eigen}
Dabei muss diese Eigenschaft nicht notwendigerweise f"ur die echte Inklusion gelten. Dazu geben wir ein Beispiel an.

\vspace{2ex}

\begin{beispiel}
Es sei $L=\{a\}, W=\{a,ab\}$ und $V=\{b\}$. Dann haben wir $L\subset W$, aber $L\ff V \not\subset W\ff W$ wegen $\emptyset \not\subset\emptyset$.
\end{beispiel}

Aus Eigenschaft \ref{fresh} folgt die Inklusion

\vspace{2ex}

\begin{eigen}
$(L\cap V) \ff W \subseteq (L\ff W) \cap (V\ff W)$
\end{eigen}
Dabei muss nicht notwendigerweise die Gleichheit gelten. Dazu geben wir wiederum ein Beispiel an.

\vspace{2ex}

\begin{beispiel}
Es seien $L= \{aa,bb\}\, ,V = \{aa,b\}\, ,W = \{aa,bb\}$. \\Dann haben wir $(L\cap V) \ff W = \{aa\} \subset \{aa,bb\} = (L\ff W) \cap (V\ff W)$
\end{beispiel}

\subsection{Monotonie im Hinterglied}
Jetzt betrachten wir die Stabilit"at der Operation Fortsetzung bezüglich $\cap$ und $\cup$ und die Monotonie bezüglich $\subseteq$ im Hinterglied.\\
Anders als bei der vorherigen Betrachtung, leiten sich in diesem Abschnitt kaum Folgerungen direkt ab.
So haben wir, im Gegensatz zu Eigenschaft \ref{cupeig}

\vspace{2ex}

\begin{eigen}\label{gl22}
$L\ff (V\cup W)\subseteq (L\ff V) \cup (L\ff W)$
\end{eigen}
Zum Beweis nutzen wir die folgende Beziehung.

\vspace{2ex}

\begin{lem}\label{lem123}
$\min (A\cup B) \subseteq \min A \cup \min B$
\end{lem}

\begin{proof}[Beweis von Lemma \ref{lem123}]
Es sei $w\in \min (A\cup B)$. Dann ist $w$ in $A$ oder in $B$ enthalten. Es sei o.B.d.A. $w\in A$. Es gilt au"serdem f"ur alle $v\sqsubset w$, dass $v\notin (A\cup B)$, also insbesondere $v\notin A$. Nach Definition von $\min$ gilt also $w\in\min(A)$. Gleiches gilt analog f"ur $B$.
\end{proof}
\begin{proof}[Beweis von Eigenschaft \ref{gl22}]
Es gilt allgemein, wenn $v\in L\ff W$, dann muss ein $w\in L$ derart existieren, dass $v\in \{w\}\ff W$. 
Daher gen"ugt es Eigenschaft \ref{gl22} f"ur $L=\{w\}$ zu zeigen.\\
Es gilt $w \ff (W\cup V)=\min (w\cdot X^* \cap (W\cup V)) =\min ((w\cdot X^* \cap W) \cup (w\cdot X^* \cap V))$\\
und mit Lemma 1 erhalten wir:\\
$ w\ff (V\cup W) \subseteq \min (w\cdot X^* \cap V) \cup \min (w\cdot X^* \cap W)=(L\ff V)\cup (L\ff W)$
\end{proof}
Dabei muss nicht notwendigerweise die Gleichheit gelten. Dazu geben wir ein Beispiel an.

\vspace{2ex}

\begin{beispiel}
Es seien $L= \{a,b\}\, ,V =\{aaa\}\, ,W = \{bb,aa\}$. \\Dann haben wir $L\ff (V\cup W)=\{aa,bb\} \subset \{aa,bb,aaa\} = (L\ff V) \cup (L\ff W)$
\end{beispiel}

Eine "ahnliche Beziehung ergibt sich f"ur die Stabilität der Operation Fortsetzung bez"uglich $\cap$.

\vspace{2ex}

\begin{eigen}\label{eigen321}
$L\ff (V\cap W)\supseteq (L\ff V) \cap (L\ff W)$
\end{eigen}
Zum Beweis nutzen wir die Beziehung.

\vspace{2ex}

\begin{lem}\label{lem321}
$\min(A\cap B) \supseteq \min A \cap \min B$
\end{lem}
%\begin{lem}
%$$L\ff U \cap L \ff V \stackrel{!}{=} \bigcup_{l\in L} (l\ff U \cap l\ff V)$$
%\end{lem}

\begin{proof}[Beweis von Lemma \ref{lem321}]
Es sei $w\in (\min(A) \cap \min(B))$. Dann ist gilt sowohl $w\in A$, als auch in $w\in B$. Au"serdem gilt f"ur alle $v$ mit $v\sqsubset w$, dass $v \notin A$ und dass $v \notin B$. Nach Definition von $\min$ gilt also $w\in \min(A\cap B)$.\emph{}
\end{proof}
\begin{proof}[Beweis von Eigenschaft \ref{eigen321}]
Es gen"ugt wiederum die Eigenschaft f"ur $L=\{w\}$ zu zeigen.\\
Es gilt $w\ff (V\cap W) = \min(w\cdot X^* \cap (V \cap W))= \min((w\cdot X^* \cap V) \cap (w\cdot X^* \cap W))$.\\
Mit Lemma 2 erhalten wir dann\\
$ w\ff (V\cap W) \supseteq \min(w\cdot X^* \cap V) \cap \min(w\cdot X^* \cap W)=(L\ff V) \cap (L\ff W)$
\end{proof}
Dabei muss nicht notwendigerweise die Gleichheit gelten. Dazu geben wir ein Beispiel an.

\vspace{2ex}

\begin{beispiel}
Es seien $L= \{a,b\}\, ,V =\{aaa,b,bb\}\, ,W = \{bb,aaa\}$\\
Dann haben wir $L\ff (V\cap W) = \{bb,aaa\} \supset \{ aaa \} =(L\ff V) \cap (L\ff W)$
\end{beispiel}

Es gilt keine Monotonie im Hinterglied: $L\subseteq W \not\to V\ff L \subseteq V \ff W$. Dazu geben wir ein Beispiel an.

\vspace{2ex}

\begin{beispiel}
Es sei $L=\{aaba\},W=\{aab,aaba\}$ und $V=\{a\}$, also $L\subseteq W$.\\
Dann haben wir $V\ff L = \{aaba\} \not\subseteq V\ff W =\{aab\}$
\end{beispiel}


%\begin{proof}
%$$w\in  (L\cap U)\ff V \to w\in V \wedge \exists p: p\sqsubseteq w \wedge p\in L \wedge p\in U \wedge p\textit{ ist minimal}$$
%$$\to \underbrace{w\in V \wedge \exists p: p\sqsubseteq w \wedge p\in L \wedge p\textit{ ist minimal}}_{w\in L\ff V} 
%\wedge \underbrace{w\in V \wedge \exists p: p\sqsubseteq w \wedge p\in U \wedge p\textit{ ist minimal}}_{w\in U\ff V} 
%$$
%\end{proof}

\section{Konkatenation}
Bisher wurde die Stabilität der Operation Fortsetzung bez"uglich der boolschen Operationen $\cap$ und $\cup$ betrachtet.
Nun betrachten wir ob es eine Stabilität der Operation Fortsetzung bezüglich der Konkatenation gibt.
Dabei werden wir feststellen, dass keine Stabiltät bezüglich der Konkatenation existiert. Dazu geben wir nun Sprachen derart an, dass jeweils alle Inklusionsbeziehungen erf"ullt werden k"onnen.\\\\
Zuerst betrachten wir die Stabilität der Operation Fortsetzung bezüglich der Konkatenation im Hinterglied.

Es existieren Sprachen $L,V$ und $W$ derart, dass $L\ff (V\cdot W) = (L\ff V) \cdot (L\ff W)$ gilt.

\vspace{2ex}

\begin{beispiel}
Es seien  $L=\{e\}\quad ,V=\{a\}\quad ,W=\{b\} $, so erhalten wir \\
$L\ff (V\cdot W) = \{ ab\} =  \{ab\} = (L\ff V) \cdot (L\ff W)$, also $L\ff (V\cdot W) = (L\ff V) \cdot (L\ff W)$
\end{beispiel}
Weiterhin existieren Sprachen derart, dass $L\ff (V\cdot W) \supset (L\ff V) \cdot (L\ff W)$ gilt.

\vspace{2ex}

\begin{beispiel}
Es seien $L=\{a\}\quad ,V=\{aa\}\quad ,W=\{b,a\} $\\
Dann erhalten wir $L\ff (V\cdot W) = \{aab,aaa\} \supset  \{aaa\} = (L\ff V) \cdot (L\ff W)$, also $L\ff (V\cdot W) \supset (L\ff V) \cdot (L\ff W)$.
\end{beispiel}
Es existieren Sprachen derart, dass $L\ff (V\cdot W) \subset (L\ff V)\cdot (L\ff W)$ gilt.

\vspace{2ex}

\begin{beispiel}
Es seien  $L=\{ab,e\}\quad ,V=\{aba,bab\}\quad ,W=\{e,aba,bab\} $, so erhalten wir \\
$L\ff (V\cdot W) = L \ff \{aba,abaaba,ababab,bab,bababa,babbab\} = \{ aba,bab\}$ und $(L\ff V)\cdot (L\ff W) = \{aba,bab\} \cdot \{e,aba\} = \{aba,abaaba,bab,bababa\}$. 
Also haben wir $L\ff (V\cdot W) \subset (L\ff V)\cdot (L\ff W)$
\end{beispiel}
Zuletzt existieren Sprachen derart, dass sowohl $L\ff (V\cdot W) \not\subset (L\ff V)\cdot (L\ff W)$ als auch $L\ff (V\cdot W) \not\supset (L\ff V)\cdot (L\ff W)$ gelten. Es gilt also eine mengentheoretische Unvergleichbarkeit.

\vspace{2ex}

\begin{beispiel}
Es seien $L=\{aab,a\}\quad V=\{aa\}\quad W=\{b\}$\\
Dann erhalten wir $L\ff (V\cdot W) = \{aab\} \not\subset \{aa\} = (L\ff V) \cdot (L\ff W)$, als auch $L\ff (V\cdot W) = \{aab\} \not\supset \{aa\} = (L\ff V) \cdot (L\ff W)$.
\end{beispiel}

Da wir f"ur alle Inklusionsbeziehungen Sprachen gefunden haben, die diese erf"ullen, kann es keine Stabilität der Operation Fortsetzung bezüglich der Konkatenation im Hinterglied geben, welche f"ur alle Sprachen $L,V,W$ gilt.\\

Betrachtet man nun die Stabilität der Operation Fortsetzung bezüglich der Konkatenation im Vorderglied, so erkennt man leicht, dass es sich 
zum Einen um W"orter aus $W$ handelt die man mit W"ortern aus $W^2$ vergleicht.\\
Demnach treten hier spezielle Inklusionsbeziehungen nur auf, wenn $W$ die Gestalt $W\subseteq W^2$ oder $W\supseteq W^2$ hat.
%\section{Gilt nicht...}
%%Folgt direkt aus den Gleichungen in 2.1
%$$L\ff (U\cup V)\supset (L\ff U) \cup (L\ff V)$$
%%$$L\ff (U\cap V)\subset (L\ff U) \cap (L\ff V)$$
%$$L\ff (U\cdot V)\subset (L\ff U) \cdot (L\ff V)$$
%$$(L\cap U) \ff V \supset (L\ff V) \cap (U\ff V)$$

%\newpage
%$$L\ff (U\cup V)\subset (L\ff U) \cup (L\ff V)$$
%$$L= \{a,b\} U =\{aaa\} V = \{bb,aa\}$$\\
%$$L\ff (U\cup V) = (L\ff U) \cup (L\ff V)$$
%$$L= \{a,b\} U =\{aaa\} V = \{aaa\}$$\\
%$$L\ff (U\cap V)\supset (L\ff U) \cap (L\ff V)$$
%$$L= \{a,b\} U =\{aaa,b,bb\} V = \{bb,aaa\}$$\\
%$$L\ff (U\cap V)= (L\ff U) \cap (L\ff V)$$
%$$L= \{a\} U =\{aaa\} V = \{aaa\}$$\\
%$$(L\cap U) \ff V = (L\ff V) \cap (U\ff V)$$
%$$L= \{aaa\} U =\{aaa\} V = \{aaa\}$$\\
%\vspace{5mm}
%$$L\ff (U\cup V) \not\supset (L\ff U) \cup (L\ff V),$$
%$$\text{Gegenbeispiel: } L=\{a\}\quad U=\{abb,aaba\}\quad V=\{aab,aba\} $$
%$$L\ff (U\cup V) = \{abb,aba,aab\}\text{ ,aber } L\ff U \cup L\ff V = \{abb,aaba\} \cup \{aab,aba\} = \{abb,aaba,aab,aba\} $$
%\vspace{5mm}
%$$L\ff (U\cap V) \not\subset (L\ff U) \cap (L\ff V),$$
%$$\text{Gegenbeispiel: } L=\{a,b\}\quad U=\{a,aa\}\quad V=\{aa,b\} $$
%$$L\ff (U\cap V) = \{aa\}\text{ ,aber } L\ff U \cap L\ff V =
%\{a\} \cap \{b\} = \emptyset $$
%\vspace{5mm}
%$$L\ff (U\cdot V)= (L\ff U) \cdot (L\ff V),$$
%$$\text{Beispiel: } L=\{e\}\quad U=\{a\}\quad V=\{b\} $$
%$$L\ff (U\cdot V) \supset (L\ff U) \cdot (L\ff V),$$
%$$\text{Beispiel: } L=\{a\}\quad U=\{aa\}\quad V=\{b,a\} $$
%\vspace{5mm}
%$$(L_1 \ff L_2) \ff L_3 \nsupseteq L_1 \ff (L_2 \ff L_3)$$
%$$\text{Gegenbeispiel: } L_1=\{ab,aa\}\quad L_2=\{a,ab\}\quad L_3=\{aa\}$$
%$$(L_1 \ff L_2) \ff L_3 = \{ab\} \ff \{aa\} = \emptyset\text{ ,aber } 
%\{ab,aa\} \ff \{aa\} = \{aa\} $$
