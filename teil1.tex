\subsection{Allgemeines}

In diesem Abschnitt betrachten wir Eigenschaften der Operation der Fortsetzung von Sprachen.
Hierbei wird zun�chst die algebraische Struktur $({2^X}^*,\ff)$ untersucht. Anschliessend werden insbesondere die Stabilit�t bzw. Monotonie bez�glich mengentheoretischen Operationen betrachtet. 

\subsection{Eigenschaften der algebraischen Struktur $({2^X}^*,\ff)$}

\subsubsection{Kommutativit�t}
Die algebraische Struktur $A=({2^X}^*,\ff)$ ist nicht kommutativ. Um dies zu zeigen, wird ein Gegenbeispiel angegeben.
\begin{beispiel}
Es seien $L=\{b\}$ und $W=\{bb\}$. Dann haben wir $L\ff W = \{bb\}$, aber $W\ff L = \emptyset$. 
\end{beispiel}
\subsubsection{Assoziativit�t}
Die algebraische Struktur $A=({2^X}^*,\ff)$ ist nicht assoziativ. Um dies zu zeigen wird wieder ein Gegenbeispiel angegeben.
\begin{beispiel}
Es seien $L=\{aa,bb\},\ V = \{aa,b\}$ und $W=\{aa,bb\}$. Dann haben wir $(L\ff V)\ff W = \{aa\} \ff W = \{aa\}$, aber $L\ff (V\ff W) = L\ff \{aa,bb\} = \{aa,bb\}$.
\end{beispiel}

\subsubsection{Einselement}
Da die algebraische Struktur $A$ nicht kommutativ ist, werden bei der Untersuchung auf Einselemente sowohl linksneutrale, als auch rechtsneutrale Einselemente betrachtet.
Wir betrachten zuerst ein linksneutrales Element $E_l$, so dass $E_l\ff L = L$ f�r alle $L\subseteq X^*$, gilt. Man erkennt leicht, dass $X^*$ ein solches Element ist, denn es gilt $X^*\ff L = L$.\\\\
Analog betrachten wir nun ein rechtsneutrales Element $E_r$, so dass $L\ff E_r = L$ f�r alle $L\subseteq X^*$, gilt. Auch hier erkennt man leicht, dass $X^*$ ein solches Element ist, denn es gilt $L\ff X^* = L$ 
\subsubsection{Nullelement}
Die Suche nach dem Nullelement stellt sich als trivial dar. Wir suchen ein Nullelement $N$, so dass $L\ff N = N$ sowie $N\ff L = N$ f�r alle $L\subseteq X^*$ gilt. Man erkennt leicht, dass die leere Menge $\emptyset$ ein solches Nullelement darstellt, da $\emptyset \ff L = \emptyset$ und $L\ff \emptyset = \emptyset$ gelten.\\\\
Damit kann die algebraische Struktur $A=({2^X}^*,\ff)$ aufgrund fehlender Assoziativit�t weder eine Gruppe noch eine Halbgruppe sein, besitzt aber Null- und Einselement.

\subsection{Boolsche Operationen}

Aus der Definition folgen direkt die Eigenschaften:

\begin{eigen}\label{eig1}
Es gilt genau dann $\{w\} \ff W = \{w\}$, wenn $w\in W$.\\
$L\cap W \subseteq L\ff W \subseteq W$
\end{eigen}

\begin{eigen}\label{cupeig}
$(L\cup V)\ff W = L\ff W\cup V\ff W$
\end{eigen}

Aus Eigenschaft \ref{eig1} wissen wir, dass $L\cap W \subseteq L\ff W$. Man kann nun die Fortsetzung von $L$ in $W$ aufsplitten, sodass sich folgende Beziehung ergibt:

\begin{eigen}\label{basiseigen}
$L\ff W = L\cap W \cup (L\backslash W) \ff W$
\end{eigen}

Aus der Eigenschaft \ref{basiseigen} folgt durch Einschr�nkung der Mengen $W$ bzw. $L$:

\begin{folg}
$L\subseteq W \leftrightarrow L\ff W = L\cap W = L$
\end{folg}
\begin{folg}
$L\supseteq W \to L\ff W = L\cap W = W$
\end{folg}


Aus Eigenschaft \ref{cupeig} folgt:
\begin{eigen}
$L\ff W \to L\ff V \subseteq W \ff V$
\end{eigen}

Damit erhalten wir die Inklusion:
\begin{gleich}
$(L\cap V) \ff W \subseteq (L\ff W) \cap (V\ff W)$
\end{gleich}
Dabei muss nicht notwendig die Gleichheit gelten, wie das folgende Beispiel zeigt
\begin{beispiel}
Es seien $L= \{aa,bb\}\, ,V = \{aa,b\}\, ,W = \{aa,bb\}$. \\Dann haben wir $(L\cap V) \ff W = \{aa\} \subset \{aa,bb\} = (L\ff W) \cap (V\ff W)$
\end{beispiel}

%\begin{proof}
%$$w\in  (L\cap U)\ff V \to w\in V \wedge \exists p: p\sqsubseteq w \wedge p\in L \wedge p\in U \wedge p\textit{ ist minimal}$$
%$$\to \underbrace{w\in V \wedge \exists p: p\sqsubseteq w \wedge p\in L \wedge p\textit{ ist minimal}}_{w\in L\ff V} 
%\wedge \underbrace{w\in V \wedge \exists p: p\sqsubseteq w \wedge p\in U \wedge p\textit{ ist minimal}}_{w\in U\ff V} 
%$$
%\end{proof}


\begin{gleich}\label{gl22}
$L\ff (V\cup W)\subseteq (L\ff V) \cup (L\ff W)$
\end{gleich}
Zum Beweis nutzen wir die folgende Beziehung.
\begin{lem}\label{lem123}
$\min (A\cup B) \subseteq \min A \cup \min B$
\end{lem}

\begin{proof}[Beweis von Lemma \ref{lem123}]
Sei $w\in \min (A\cup B)$. Dann ist $w$ in $A$ oder in $B$ enthalten. OBdA sei $w\in A$. Es gilt au�erdem f�r alle $v\sqsubset w$, dass $v\notin (A\cup B)$, also insbesondere $v\notin A$. Nach Definition von $\min$ gilt also $w\in\min(A)$. Gleiches gilt analog, wenn man annimmt $w\in B$.
\end{proof}
\begin{proof}[Beweis von Gleichung \ref{gl22}]
Es gen�gt, die Eigenschaft f�r $L=\{w\}$ zu zeigen.\\
Es gilt $w \ff (W\cup V)=\min (w\cdot X^* \cap (W\cup V)) =\min ((w\cdot X^* \cap W) \cup (w\cdot X^* \cap V))$\\
und mit Lemma 1 erhalten wir:\\
$ w\ff (V\cup W) \subseteq \min (w\cdot X^* \cap V) \cup \min (w\cdot X^* \cap W)=(L\ff V)\cup (L\ff W)$
\end{proof}
Dabei muss nicht notwendig die Gleichheit gelten, wie das folgende Beispiel zeigt.
\begin{beispiel}
Es seien $L= \{a,b\}\, ,V =\{aaa\}\, ,W = \{bb,aa\}$. \\Dann haben wir $L\ff (V\cup W)=\{aa,bb\} \subset \{aa,bb,aaa\} = (L\ff V) \cup (L\ff W)$
\end{beispiel}

Eine �hnliche Beziehung ergibt sich f�r die Operation $\cup$

\begin{gleich}\label{eigen321}
$L\ff (V\cap W)\supseteq (L\ff V) \cap (L\ff W)$
\end{gleich}
Zum Beweis bem�hen wir die Beziehung aus 
\begin{lem}\label{lem321}
$\min(A\cap B) \supseteq \min A \cap \min B$
\end{lem}
%\begin{lem}
%$$L\ff U \cap L \ff V \stackrel{!}{=} \bigcup_{l\in L} (l\ff U \cap l\ff V)$$
%\end{lem}

\begin{proof}[Beweis von Lemma \ref{lem321}]
Sei $w\in (\min(A) \cap \min(B))$. Dann ist $w$ sowohl in $A$, als auch in $B$ enthalten. Au�erdem gilt f�r alle $v$ mit $v\sqsubset w$, dass $v \notin A$ und dass $v \notin B$. Nach Definition von $\min$ gilt also $w\in \min(A\cap B)$.\emph{}
\end{proof}
\begin{proof}[Beweis von Eigenschaft \ref{eigen321}]
Es gen�gt die Eigenschaft f�r $L=\{w\}$ zu zeigen.\\
Es gilt $w\ff (V\cap W) = \min(w\cdot X^* \cap (V \cap W))= \min((w\cdot X^* \cap V) \cap (w\cdot X^* \cap W))$\\
und mit Lemma 2 erhalten wir\\
$ w\ff (V\cap W) \supseteq \min(w\cdot X^* \cap V) \cap \min(w\cdot X^* \cap W)$\\
$=(L\ff V) \cap (L\ff W)$
\end{proof}
Dabei muss nicht notwendig die Gleichheit gelten, wie das folgende Beispiel zeigt
\begin{beispiel}
Es seien $L= \{a,b\}\, ,V =\{aaa,b,bb\}\, ,W = \{bb,aaa\}$\\
Dann haben wir $L\ff (V\cap W) = \{bb,aaa\} \supset \{ aaa \} =(L\ff V) \cap (L\ff W)$
\end{beispiel}

\newpage
\subsection{Konkatenation}
Bisher wurden haupts�chlich die Operationen $\cap$ sowie $\cup$ in Verbindung mit $\ff$ betrachtet.
F�r die Konkatenation ergeben sich keine Eigenschaften allgemeing�ltiger Natur.\\

So gibt es Sprachen, bei denen die Gleichheit in folgender Weise gegeben ist:
\begin{beispiel}
Es seien  $L=\{e\}\quad ,V=\{a\}\quad ,W=\{b\} $, so erhalten wir \\
$L\ff (V\cdot W) = \{ ab\} =  \{ab\} = (L\ff V) \cdot (L\ff W)$, also $L\ff (V\cdot W) = (L\ff V) \cdot (L\ff W)$
\end{beispiel}
weiterhin k�nnen wir Sprachen angeben, sodass Teilmengenbeziehung in folgender Weise existieren:
\begin{beispiel}
Es seien $L=\{a\}\quad ,V=\{aa\}\quad ,W=\{b,a\} $\\
Dann erhalten wir $L\ff (V\cdot W) = \{aab,aaa\} \supset  \{aaa\} = (L\ff V) \cdot (L\ff W)$, also $L\ff (V\cdot W) \supset (L\ff V) \cdot (L\ff W)$.
\end{beispiel}

Das letzte Beispiel in diesem Abschnitt zeigt deutlich, dass keine allgemeing�ltigen Eigenschaften f�r die Operation $\cdot$ bez�glich $\ff$ existieren:
\begin{beispiel}
Es seien $L=\{aab,a\}\quad V=\{aa\}\quad W=\{b\}$\\
Dann erhalten wir $L\ff (V\cdot W) = \{aab\} \neq \{aa\} = (L\ff V) \cdot (L\ff W)$, das bedeutet also, dass beide Seiten mengentheoretisch unvergleichbar sind, also 
$L\ff (V\cdot W) \not\supset (L\ff V) \cdot (L\ff W)$, sowie $L\ff (V\cdot W) \not\subset (L\ff V) \cdot (L\ff W)$
\end{beispiel}
Betrachtet man nun die Konkatenation `vorn`, also die Beziehung $(L\cdot V)\ff V$ zu  $(L\ff W) \cdot (V\ff W)$, so sieht man leicht, dass es sich auf 
der linken Seite der Gleichung um W�rter aus $W$ handelt die man vergleicht mit W�rtern aus $W^2$. \\Demnach treten hier Eigenschaften nur auf wenn $W$ eine ganz spezielle Gestalt hat.
%\subsection{Gilt nicht...}
%%Folgt direkt aus den Gleichungen in 2.1
%$$L\ff (U\cup V)\supset (L\ff U) \cup (L\ff V)$$
%%$$L\ff (U\cap V)\subset (L\ff U) \cap (L\ff V)$$
%$$L\ff (U\cdot V)\subset (L\ff U) \cdot (L\ff V)$$
%$$(L\cap U) \ff V \supset (L\ff V) \cap (U\ff V)$$

%\newpage
%$$L\ff (U\cup V)\subset (L\ff U) \cup (L\ff V)$$
%$$L= \{a,b\} U =\{aaa\} V = \{bb,aa\}$$\\
%$$L\ff (U\cup V) = (L\ff U) \cup (L\ff V)$$
%$$L= \{a,b\} U =\{aaa\} V = \{aaa\}$$\\
%$$L\ff (U\cap V)\supset (L\ff U) \cap (L\ff V)$$
%$$L= \{a,b\} U =\{aaa,b,bb\} V = \{bb,aaa\}$$\\
%$$L\ff (U\cap V)= (L\ff U) \cap (L\ff V)$$
%$$L= \{a\} U =\{aaa\} V = \{aaa\}$$\\
%$$(L\cap U) \ff V = (L\ff V) \cap (U\ff V)$$
%$$L= \{aaa\} U =\{aaa\} V = \{aaa\}$$\\
%\vspace{5mm}
%$$L\ff (U\cup V) \not\supset (L\ff U) \cup (L\ff V),$$
%$$\text{Gegenbeispiel: } L=\{a\}\quad U=\{abb,aaba\}\quad V=\{aab,aba\} $$
%$$L\ff (U\cup V) = \{abb,aba,aab\}\text{ ,aber } L\ff U \cup L\ff V = \{abb,aaba\} \cup \{aab,aba\} = \{abb,aaba,aab,aba\} $$
%\vspace{5mm}
%$$L\ff (U\cap V) \not\subset (L\ff U) \cap (L\ff V),$$
%$$\text{Gegenbeispiel: } L=\{a,b\}\quad U=\{a,aa\}\quad V=\{aa,b\} $$
%$$L\ff (U\cap V) = \{aa\}\text{ ,aber } L\ff U \cap L\ff V =
%\{a\} \cap \{b\} = \emptyset $$
%\vspace{5mm}
%$$L\ff (U\cdot V)= (L\ff U) \cdot (L\ff V),$$
%$$\text{Beispiel: } L=\{e\}\quad U=\{a\}\quad V=\{b\} $$
%$$L\ff (U\cdot V) \supset (L\ff U) \cdot (L\ff V),$$
%$$\text{Beispiel: } L=\{a\}\quad U=\{aa\}\quad V=\{b,a\} $$
%\vspace{5mm}
%$$(L_1 \ff L_2) \ff L_3 \nsupseteq L_1 \ff (L_2 \ff L_3)$$
%$$\text{Gegenbeispiel: } L_1=\{ab,aa\}\quad L_2=\{a,ab\}\quad L_3=\{aa\}$$
%$$(L_1 \ff L_2) \ff L_3 = \{ab\} \ff \{aa\} = \emptyset\text{ ,aber } 
%\{ab,aa\} \ff \{aa\} = \{aa\} $$
