%einleitung
In dieser Arbeit untersuchen wir die Operation Fortsetzung bei formalen Sprachen.
Diese Operation wird in der Arbeit [St87] eingef�hrt. Dabei bezeichnen wir die Fortsetzung eines Wortes $w$ in eine Sprache $L$, als das Minimum aller W�rter aus $L$, in denen $w$ ein Pr�fix ist. 
\\Man stelle sich einen Ableitungsbaum vor beginnend bei $w$.
Man folgt nun allen Pfaden von $w$ nach $X^*$. Trifft man auf einem Pfad auf ein Wort aus $L$, so wird dieses Wort dem Ergebnis hinzugef�gt und diesem Pfad wird nicht mehr gefolgt.\\
Wir bezeichnen die Fortsetzung einer Sprache $L$ in eine Sprache $W$ als Vereinigung der Fortsetzungen aller W�rter aus $L$ in $W$.\\\\
Wie in der Arbeit [St87] behandelt, erleichtert die Operation Fortsetzung den Schnitt beim $\delta$-Limes zweier Sprachen, mehr dazu im zweiten Abschnitt dieser Arbeit.\\\\
In dieser Arbeit wird deshalb die Operation Fortsetzung untersucht. Zun�chst legen wir die verwendete Notation fest. Dann betrachten wir die algebraische Struktur $({2^X}^*,\ff)$ und untersuchen diese auf typische Eigenschaften.
Da diese Struktur nicht kommutativ ist, untersuchen wir dann die Monotonie der Operation Fortsetzung in Bezug auf die mengentheoretischen Operationen $\cap,\cup$ und der Konkatenation im Vorder- sowie Hinterglied.
\\Im dann folgenden Abschnitt untersuchen wir die Operation Fortsetzung, wenn eine der beiden Operanden die spezielle Form $\pref(L)$ oder $W\cdot X^*$ hat. Abschlie�end betrachten wir die Abgeschlossenheitseigenschaften der Operation Fortsetzung in der CHOMSKY-Hierachie.



