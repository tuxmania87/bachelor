%einleitung
In dieser Arbeit untersuchen wir die Operation Fortsetzung f"ur formale Sprachen, welche in der Arbeit [St87] eingef"uhrt wurde. Dabei bezeichnen wir die Fortsetzung eines Wortes $w$ in eine Sprache $L$, als das Minimum aller W"orter aus $L$, in denen $w$ ein Pr"afix ist. 
\\Man stelle sich einen Ableitungsbaum vor, beginnend bei $w$.
Man folgt nun allen Pfaden von $w$ nach $X^*$. Trifft man auf einem Pfad auf ein Wort aus $L$, so wird dieses Wort dem Ergebnis hinzugef"ugt und diesem Pfad wird nicht mehr gefolgt.\\
Wir definieren die Fortsetzung einer Sprache $L$ in eine Sprache $W$ als Vereinigung der Fortsetzungen aller W"orter aus $L$ in $W$.\\\\
Wie in der Arbeit [St87] behandelt, erleichtert die Operation Fortsetzung die Durchschnittsbildung beim $\delta$-Limes zweier Sprachen, mehr dazu im zweiten Abschnitt dieser Arbeit.\\\\
Zuerst legen wir die verwendete Notation fest. Dann betrachten wir die algebraische Struktur $({2^X}^*,\ff)$ und untersuchen diese auf typische Eigenschaften.
Da diese Struktur nicht kommutativ ist, untersuchen wir dann die Stabilit"at der Operation Fortsetzung in Bezug auf die mengentheoretischen Operationen $\cap,\cup$ und der Konkatenation im Vorder- sowie Hinterglied.
Au"serdem wird in diesem Abschnitt die Monotonie der Operation Fortsetzung bezüglich $\subseteq$ betrachtet.
\\Im dann folgenden Abschnitt untersuchen wir die Operation Fortsetzung, wenn eine der beiden Operanden die spezielle Form $\pref(L)$ oder $W\cdot X^*$ hat. Abschlie"send betrachten wir die Abgeschlossenheitseigenschaften der Operation Fortsetzung f"ur die Klassen der CHOMSKY-Hierachie.



