%einleitung
In dieser Arbeit untersuchen wir die Operation Fortsetzung bei formalen Sprachen.
Diese Operation wird in der Arbeit [St87] eingef�hrt.\\\\
Wir definieren den $\delta$-Limes einer Wortmenge $W ^\delta$ mit (s. [St87, Seite X])
\begin{def1}
$W^\delta = \{\beta : \beta \in X^\omega \textit{ und }\pref(\beta)\cap W \textit{ ist unendlich}\}$
\end{def1}
% also eine unendliche Aneinanderreihung von W�rtern aus ....
Die folgende Eigenschaft (13) aus [St87] ist leich einzusehen:
$(U \cup W)^\delta = U^\delta \cup W^\delta $
	\begin{def1}
	Eine Sprache nennen wir eine $(\sigma,\delta)$-Teilmenge von $X^*$ genau dann, wenn f�r alle $\beta \in X^\omega$ entweder $\pref(\beta)\cap W$ oder $\pref(\beta)\backslash W$ endlich ist.	
	\end{def1}
Beispiele f�r $(\sigma,\delta)$-Teilmengen sind alle endlichen Sprachen und deren Komplemente. Weitere Beispiele sind Sprachen der Form $\pref(U)$ oder $W\cdot X^*$.\\
Eine Eigenschaft f�r diese Teilmengen ergibt sich wiefolgt :
\begin{satz}[St87]
$\text{Sei }U\text{ eine }(\sigma,\delta)-\text{Teilmenge von }X^*, \text{ dann gilt:}$\\
$(U\cap W)^\delta = U^\delta \cap W^\delta,\qquad \text{f�r alle }W\subseteq X^*$
\end{satz}

Nun wird die Operation "'Fortsetzung"' wie in [St87] eingef�hrt, im nachfolgenden als $\ff$ bezeichnet.
Die Fortsetzung eines Wortes $w$ in eine Sprache $V\subseteq X^*$ sei definiert als:
\begin{def1}
$w\ff V := \min_{\sqsubseteq} \{v:v\in V \wedge w\sqsubseteq v\} = \min(w\cdot X^* \cap V)$
\end{def1}
Diese Operation wird wie folgt auf Sprachen ausgedehnt, dabei bezeichnen wir die Fortsetzung einer Sprache W in eine Sprache $V\subseteq X^*$ mit:
\begin{def1}\label{continuecup}
$W\ff V := \bigcup_{w\in W} w\ff V$
\end{def1}
Diese Operation hat nun folgende Eigenschaft bez�glich des $\delta$-Limes:
W�hrend $$(W\cap U)^\delta = W^\delta \cap U^\delta$$ nur f�r $(\sigma,\delta)$-Teilemgen gilt, so gilt
$$(W\ff U)^\delta = W^\delta \cap U^\delta$$ f�r s�mtliche Sprachen\\
Daher wird nun im Verlauf der Arbeit die Operation Fortsetzung untersucht.
