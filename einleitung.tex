%einleitung
In dieser Arbeit wird die Operation Fortsetzung bei formalen Sprachen untersucht.
Im Folgenden wird erl�utert warum die Operation in [St87] eingef�hrt wurde und welchen Nutzen sie birgt.\\\\
Wir bezeichnen die Menge $X^*$ als Menge aller endlichen W�rter �ber dem Alphabet $X$. 
\\Wir bezeichnen weiterhin die Menge $X^\omega$ als Menge aller unendlichen W�rter �ber dem Alphabet $X$.
\\Sei ferner die Relation $\sqsubseteq$ wie �blicherweise definiert:
\begin{def1}
$$w\sqsubseteq b \Leftrightarrow w\cdot b' = b,\textit{ f{\"u}r ein }b'\in X^*$$
$$\pref(L) = \{v:v\sqsubseteq w \wedge w\in L\}$$
\end{def1}
Sei nun $W ^\delta$ definiert \footnote{St87]}
\begin{def1}
$$W^\delta = \{\beta : \beta \in X^\omega \textit{ und }\pref(\beta)\cap W \textit{ ist unendlich}\}$$
\end{def1}
% also eine unendliche Aneinanderreihung von W�rtern aus ....
Folgende Eigenschaft wurde nun bereits in \footnote{[St87, Gleichung 13]} bewiesen:
$$(U \cup W)^\delta = U^\delta \cup W^\delta $$
	\begin{def1}
	Eine Sprache nennen wir eine $(\sigma,\delta)$-Teilmenge von $X^*$ genau dann, wenn f�r alle $\beta \in X^\omega$ entweder $\pref(\beta)\cap W$ oder $\pref(\beta)\backslash W$ endlich ist.	
	\end{def1}
Beispiele f�r $(\sigma,\delta)$-Teilmengen sind alle endlichen Sprachen und deren Komplemente. Weitere Beispiele sind Sprachen der Form $\pref(U)$ oder $W\cdot X^*$.\\
Eine Eigenschaft f�r diese Teilmengen lautet wiefolgt:
\begin{satz}
$\text{Sei }U\text{ eine }(\sigma,\delta)-\text{Teilmenge von }X^*, \text{ dann gilt:}$\\
$$(U\cap W)^\delta = U^\delta \cap W^\delta,\qquad \text{f�r alle }W\subseteq X^*$$
\end{satz}

Nun wird die Operation "'Fortsetzung"' eingef�hrt \footnote{[St87, S.170]}, im nachfolgenden als $\ff$ bezeichnet.
Die Fortsetzung eines Wortes $w$ in $V$ sei definiert als:
\begin{def1}
$$w\ff V := \min_{\sqsubseteq} \{v:v\in V \wedge w\sqsubseteq v\} = \min(w\cdot X^* \cap V)$$
\end{def1}
Man kann diese Definition nun ausdehnen auf Sprachen. Die Fortsetzung zweier Sprachen W und V ergibt sich somit zu:
\begin{def1}
$$W\ff V := \bigcup_{w\in W} w\ff V$$
\end{def1}
Diese Operation hat nun folgende interessante Eigenschaft bez�glich der oben genannten Definitionen:
W�hrend $$(W\cap U)^\delta = W^\delta \cap U^\delta$$ nur f�r $(\sigma,\delta)$-Teilemgen gilt, so gilt aber
$$(W\ff U)^\delta = W^\delta \cap U^\delta$$ f�r s�mtliche Sprachen \footnote{[St87, Gleichung 20]}\\
Daher wird nun im Verlauf der Arbeit die Operation Fortsetzung gr�ndlich analysiert und all ihre Eigenschaften dokumentiert.
