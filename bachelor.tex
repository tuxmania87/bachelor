
\documentclass[11pt]{scrartcl}
\usepackage{geometry}
\usepackage{graphics}
\usepackage{graphicx}
\usepackage{amssymb}
\usepackage{amstext}
\usepackage{amsmath}
\usepackage{amsthm}
\usepackage{color}
\usepackage{amsmath}
\usepackage[ngerman]{babel}
\usepackage[latin1]{inputenc}
\usepackage{units}
\usepackage{ulem}
\usepackage{algorithm}
\usepackage{algorithmic}
%\geometry{a4paper,left=20mm,right=10mm, top=13mm, bottom=20mm}
\newcommand{\ff}{\triangleright}
\newcommand{\pref}[1]{\mathit{pref{#1}}}
\author{Robert Hartmann}
\title{�ber die Operation Fortsetzung bei formalen Sprachen}
\date{24. September 2010}
\parindent 0pt
\begin{document}

\newtheorem{def1}{Definition} 
\newtheorem{satz}{Satz}
\newtheorem{lem}{Lemma}  
\newtheorem{bew}{Beweis} 
\newtheorem{gleich}{Gleichung}
\newtheorem{eigen}{Eigenschaft}

\maketitle
\newpage
\tableofcontents
\newpage
\section{Einleitung}
%einleitung
In dieser Arbeit untersuchen wir die Operation Fortsetzung f"ur formalen Sprachen.
Diese Operation wird in der Arbeit [St87] eingef"uhrt. Dabei bezeichnen wir die Fortsetzung eines Wortes $w$ in eine Sprache $L$, als das Minimum aller W"orter aus $L$, in denen $w$ ein Pr"afix ist. 
\\Man stelle sich einen Ableitungsbaum vor beginnend bei $w$.
Man folgt nun allen Pfaden von $w$ nach $X^*$. Trifft man auf einem Pfad auf ein Wort aus $L$, so wird dieses Wort dem Ergebnis hinzugef"ugt und diesem Pfad wird nicht mehr gefolgt.\\
Wir definieren die Fortsetzung einer Sprache $L$ in eine Sprache $W$ als Vereinigung der Fortsetzungen aller W"orter aus $L$ in $W$.\\\\
Wie in der Arbeit [St87] behandelt, erleichtert die Operation Fortsetzung den Schnitt beim $\delta$-Limes zweier Sprachen, mehr dazu im zweiten Abschnitt dieser Arbeit.\\\\
In dieser Arbeit wird deshalb die Operation Fortsetzung untersucht. Zun"achst legen wir die verwendete Notation fest. Dann betrachten wir die algebraische Struktur $({2^X}^*,\ff)$ und untersuchen diese auf typische Eigenschaften.
Da diese Struktur nicht kommutativ ist, untersuchen wir dann die Stabilit"at der Operation Fortsetzung in Bezug auf die mengentheoretischen Operationen $\cap,\cup$ und der Konkatenation im Vorder- sowie Hinterglied.
\\Im dann folgenden Abschnitt untersuchen wir die Operation Fortsetzung, wenn eine der beiden Operanden die spezielle Form $\pref(L)$ oder $W\cdot X^*$ hat. Abschlie"send betrachten wir die Abgeschlossenheitseigenschaften der Operation Fortsetzung f"ur die Klassen der CHOMSKY-Hierachie.




\newpage

\section{allgemeine Eigenschaften}
\subsection{Gilt f�r alle Sprachen}
$$u\in L \to (u\ff L=\{u\})$$
$$L\ff L = L$$
$$U\ff L \subseteq L$$
$$L\ff (U\cup V)\subseteq (L\ff U) \cup (L\ff V)$$
$$L\ff (U\cap V)\supseteq (L\ff U) \cap (L\ff V)$$
$$(L\cup U)\ff V = (L\ff V) \cup (U\ff V)$$
$$(L\cap U) \ff V \subseteq (L\ff V) \cap (U\ff V)$$


\subsection{Gilt f�r einige Sprachen $\exists L,U,V \subseteq X^*:$}
$$L_2\subseteq L_1 \to L_2\ff W \subseteq L_1 \ff W$$
$$\xout{W_2\subseteq W_1 \to L\ff W_2 \subseteq L \ff W_1}$$
$$L_1\subseteq L_2 \to L_1\ff L_2 = L_1$$
$$L_2\subseteq L_1 \to L_1\ff L_2 = L_2$$
$$L\ff (U\cdot V) = (L\ff U) \cdot (L\ff V)$$
$$L\ff (U\cdot V) \supset (L\ff U) \cdot (L\ff V)$$


\subsection{Gilt nicht...}
$$L\ff (U\cup V)\supset (L\ff U) \cup (L\ff V)$$
$$L\ff (U\cap V)\subset (L\ff U) \cap (L\ff V)$$
$$L\ff (U\cdot V)\subset (L\ff U) \cdot (L\ff V)$$
$$(L\cap U) \ff V \supset (L\ff V) \cap (U\ff V)$$

\newpage
$$L\ff (U\cup V)\subset (L\ff U) \cup (L\ff V)$$
$$L= \{a,b\} U =\{aaa\} V = \{bb,aa\}$$\\
$$L\ff (U\cup V) = (L\ff U) \cup (L\ff V)$$
$$L= \{a,b\} U =\{aaa\} V = \{aaa\}$$\\
$$L\ff (U\cap V)\supset (L\ff U) \cap (L\ff V)$$
$$L= \{a,b\} U =\{aaa,b,bb\} V = \{bb,aaa\}$$\\
$$L\ff (U\cap V)= (L\ff U) \cap (L\ff V)$$
$$L= \{a\} U =\{aaa\} V = \{aaa\}$$\\
$$(L\cap U) \ff V \subset (L\ff V) \cap (U\ff V)$$
$$L= \{aa,bb\} U = \{aa,b\} V = \{aa,bb\}$$\\
$$(L\cap U) \ff V = (L\ff V) \cap (U\ff V)$$
$$L= \{aaa\} U =\{aaa\} V = \{aaa\}$$\\
\vspace{5mm}
$$L\ff (U\cup V) \not\supset (L\ff U) \cup (L\ff V),$$
$$\text{Gegenbeispiel: } L=\{a\}\quad U=\{abb,aaba\}\quad V=\{aab,aba\} $$
$$L\ff (U\cup V) = \{abb,aba,aab\}\text{ ,aber } L\ff U \cup L\ff V = \{abb,aaba\} \cup \{aab,aba\} = \{abb,aaba,aab,aba\} $$
\vspace{5mm}
$$L\ff (U\cap V) \not\subset (L\ff U) \cap (L\ff V),$$
$$\text{Gegenbeispiel: } L=\{a,b\}\quad U=\{a,aa\}\quad V=\{aa,b\} $$
$$L\ff (U\cap V) = \{aa\}\text{ ,aber } L\ff U \cap L\ff V =
\{a\} \cap \{b\} = \emptyset $$
\vspace{5mm}
$$L\ff (U\cdot V)= (L\ff U) \cdot (L\ff V),$$
$$\text{Beispiel: } L=\{e\}\quad U=\{a\}\quad V=\{b\} $$
$$L\ff (U\cdot V) \supset (L\ff U) \cdot (L\ff V),$$
$$\text{Beispiel: } L=\{a\}\quad U=\{aa\}\quad V=\{b,a\} $$
\vspace{5mm}
$$(L_1 \ff L_2) \ff L_3 \nsupseteq L_1 \ff (L_2 \ff L_3)$$
$$\text{Gegenbeispiel: } L_1=\{ab,aa\}\quad L_2=\{a,ab\}\quad L_3=\{aa\}$$
$$(L_1 \ff L_2) \ff L_3 = \{ab\} \ff \{aa\} = \emptyset\text{ ,aber } 
\{ab,aa\} \ff \{aa\} = \{aa\} $$
\vspace{5mm}
$$L\cdot X^* \ff X^* = \bigcup_{u\in L\cdot X^*} u\ff X^* = \bigcup_{u\in L\cdot X^*} \min_{\sqsubseteq} \{l:l\in X^*\wedge u\sqsubseteq L\}$$ 
$$= \bigcup_{u\in L\cdot X^*} \min(X^* \cap u\cdot X^*)=\bigcup_{u\in L\cdot X^*} \min(u) = \bigcup_{u\in L\cdot X^*} u = L\cdot X^*$$\\

$$L\cdot X^* \ff L = \bigcup_{u\in L\cdot X^*} u\ff L = \bigcup_{u\in L\cdot X^*} \min_{\sqsubseteq} \{l:l\in L \wedge u\sqsubseteq L\} = \bigcup_{u\in L\cdot X^*} \min(l\cap u\cdot X^*) $$
1. Fall: $u\in L \Rightarrow l\cap u\cdot X^* = u$\\
2. Fall: $u\in  L\cdot X^+ \Rightarrow l\cap u\cdot X^* = \emptyset$ oder $l\cap u\cdot X^* = u$ wenn $u\in L$\\
$$\bigcup_{u\in L\cdot X^*} \min(u) = L$$


%umformungen
\newpage
Lemma: $\min (A\cup B) \subseteq \min A \cup \min B$
$$L\ff (W\cup V)= \bigcup_{l\in L} l\ff (W\cup V)$$
$$=\bigcup_{l\in L} \min (l\cdot X^* \cap (W\cup V))$$
$$=\bigcup_{l\in L} \min ((l\cdot X^* \cap W) \cup (l\cdot X^* \cap V))$$
$$\subseteq \bigcup_{l\in L} (\min (l\cdot X^* \cap W) \cup \min (l\cdot X^* \cap V))$$
$$=\bigcup_{l\in L} \min (l\cdot X^*\cap W) \cup \bigcup_{l\in L} \min(l\cdot X^* \cap V)$$
$$=(L\ff W)\cup (L\ff V)$$
\vspace{5mm}
$$(L\cup U) \ff V = \bigcup_{l\in L\cup U} \min(l\cdot X^* \cap V)$$
$$ =\bigcup_{l\in L} \min(l\cdot X^* \cap V) \cup \bigcup_{l\in U} \min(l\cdot X^* \cap V)$$
$$ = (L\ff V) \cup (U\ff V)$$

\newpage

\section{Eigenschaften bei Sprachen spezieller Gestalt}

\underline{$\textit{pref}(V)\ff W$}\\
1. Fall: $w\in W \cap pref(V) \to w\in \textit{pref}(V)\ff W$\\
2. Fall: $w\in W\backslash \textit{pref}(V) \to $muss im Einzelfall gepr�ft werden.\\
$\to pref(V)\ff W = \underbrace{W\cap \textit{pref}(V)}_A \cup  \textit{pref(V)}\backslash A \ff W$\\\\


\underline{$W\ff \textit{pref}(V) = W \cap \textit{pref}(V)$}\\
"'$\rightarrow$"'\\
$\bigcup_{w\in W} \min_{\sqsubseteq} \{v:v\in pref(V) \wedge w\sqsubseteq v\}$\\
$v\in pref(V) \wedge w\sqsubseteq v \to w\in pref(V)$\\
$\bigcup_{w\in W} \{w\in pref(V)\} = W\cap \textit{pref}(V)$\\
\\
%"'$\leftarrow$"'\\
%$w\in \textit{pref}(V) \wedge w\in W \to w\sqsubseteq v\in \textit{pref}(V)$\\\\
\underline{$V\cdot X^* \ff W$}\\
$\bigcup_{v\in VX^*} \min_{\sqsubseteq} \{w:w\in W \wedge v\sqsubseteq w\}$\\
$v\in VX^* \wedge v\sqsubseteq w \to w\in VX^*$\\
$=\{w:w\in W \wedge w\in VX^*\} = W\cap VX^*$\\\\
%$=\bigcup_{v\in V} \{w:w\in W \wedge v\sqsubseteq w\}$\\\\
\underline{$W\ff V\cdot X^*$}\\
1.Fall: $w\in W\wedge w\in VX^* \to \{w\} \ff VX^* = \{w\} \cap VX^*$\\
1.Fall: $w\in W\wedge w\notin VX^* \to \{w\} \ff VX^* = \emptyset$\\

\newpage
$$ V = \{v:\forall w(w\in W \to w\not\sqsubseteq v)\}\qquad \mathit{pref}(V)=\{u:\exists v (u\sqsubseteq v\wedge v\in V)\}$$\\

$$W\ff V = \emptyset$$
$$W\ff \mathit{pref}(V) = \emptyset$$
$$W\cdot X^* \ff V = \emptyset$$
$$W\cdot X^* \ff \mathit{pref}(V) = \emptyset$$
$$\mathit{pref}(V)\ff W\cdot X^* = $$

$$W\ff V = \bigcup_{x\in W} x\ff V = \bigcup \min_{\sqsubseteq } \{v:v\in V \wedge x\sqsubseteq v\}$$
Wegen $x\in W$ und $x\sqsubseteq v$ gilt $\{v:v\in V \wedge x\sqsubseteq v\} = \emptyset \Rightarrow \bigcup \min_{\sqsubseteq} \{v:v\in V \wedge x\sqsubseteq v\} = \emptyset$\\\\
$$ W\cdot X^* \ff V = \bigcup_{x\in W\cdot X^*} x\ff V = \bigcup_{x\in W\cdot X^*} \min_{\sqsubseteq} \{v:v\in V \wedge x\sqsubseteq v\} = \emptyset$$
Begr�ndung analog Fall $W\ff V$\\\\
$$ W \ff \mathit{pref}(V) = \bigcup_{w\in W} w\ff \mathit{pref}(V) = \bigcup_{w\in W} \{v:v\in \mathit{pref}(V) \wedge w\sqsubseteq v\} = \emptyset$$
$w\sqsubseteq v$ ist nie erfuellt. Angenommen $v\in \mathit{pref}(V) \wedge w\sqsubseteq v$, dann kann man $v$ wie folgt verlaengern $v'=v\cdot v''$ mit $v'\in V$, dann waere $v'=w\cdot w' \cdot v''$ mit $w\cdot w' = v$.
Dadurch gilt aber $v'\in V\wedge w\sqsubseteq v' \wedge w\in W \Rightarrow$ Widerspruch zur Definition.\\\\

$$\mathit{pref}(V) \ff W\cdot X^* = \bigcup_{u\in \mathit{pref}(V)} u\ff W\cdot X^* = \bigcup_{u\in \mathit{pref}(V)} \min_{\sqsubseteq} \{w:w\in W\cdot X^* \wedge u\sqsubseteq w\}$$
$$ =  \bigcup_{u\in \mathit{pref}(V)} \min(W\cdot X^* \cap u\cdot X^*) =  \bigcup_{u\in \mathit{pref}(V)} \min(\{w:w\in W \wedge u\sqsubseteq w\})$$\\
$$W\cdot X^* \ff \mathit{pref}(V) = \bigcup_{u\in W\cdot X^*} u\ff \mathit{pref}(V) = \bigcup_{u\in W\cdot X^*} \min_{\sqsubseteq} \{w:w\in \mathit{pref}(V) \wedge u\sqsubseteq w\}$$
$$ = \bigcup_{u\in W\cdot X^*} \min(\mathit{pref}(V)\cap u\cdot X^*)$$
$\mathit{pref}(V)\cap u\cdot X^* = \emptyset$, Begr�ndung siehe oben $\Rightarrow  \bigcup_{u\in W\cdot X^*} \min(\emptyset) = \emptyset$

\newpage

\section{Abgeschlossenheit in der CHOMSKY-Hierachie}
\subsection{Regularit�t}
Seien $L$ und $W$ regul\"ar, so ist auch $L\ff W$ regul\"ar.\\\\
Automat $A_L= (X,Z,z_{0},\delta_L,Z_f)$ akzeptiere $L$, Automat $A_W = (X,S,s_{0},f,S_{f})$ akzeptiere $W$.
Automat $A$ akzeptiert $L\ff W$, \\\\Vorgehensweise:\\
$A_L$ und $A_W$ lesen das Wort $w$ parallel. Falls $A_L$ akzeptiert und w�hlt $A$ nicht-deterministisch aus ob Schritt 2 aktiviert wird oder nicht.\\
Schritt 2: $A_W$ liest das Wort $w$ zu Ende, w\"ahrend $A_L$ im Zustand $z_f'$ verweilt. Sollte $A_W$ auf diesem mehr als einmal akzeptieren, so akzeptiert $A$ nicht indem $A_W$ im Stoppzustand $s_x$ stehen bleibt, ansonsten akzeptiert $A$.\\

$A = (X,Z\cup \{ z_f'\}\times S\cup \{ s_x\}, (z_{0},s_{0} ), \delta , \{ (z_f',s') : s'\in S_f \} ), s_x\notin S$
mit\\\\
$\delta = \{ ((z_i,s_i),x,(z_j,s_j)) : (z_i,x,z_j) \in \delta_L \wedge f(s_i,x)=s_j  \} \cup \\
 \{ ((z_i,s_i),x,(z_f',s_j)) : (z_i,x,z') \in \delta_L \wedge z'\in Z_f \wedge f(s_i,x)=s_j  \} \cup \\
 \{ ((z_f',s_i),x,(z_f',s_j)) : f(s_i,x)=s_j  \wedge s_i \notin S_f\} \cup  \\
 \{ ((z_f',s_i),x,(z_f',s_x)) : f(s_i,x)=s_j  \wedge s_i \in S_f\} 
 $\\\\

\subsection{Kontextfreiheit}

\subsubsection{deterministisch kontextfrei}

Es existieren deterministisch kontextfreie Sprachen $L,W$, sodass $L\ff W$ nicht deterministisch kontextfrei ist!

$$L=\{a^nb^nc^i:i,n>0\}\qquad W=\{a^ib^nc^n:i,n>0\}$$
So ist $$L\ff W = \bigcup_{l\in L} \min_{\sqsubseteq}\{w:w\in W \wedge l\sqsubseteq w\} = \{a^nb^nc^n:n>0\} = U$$
Und von $U$ wissen wir, dass es nicht kontextfrei, also auch nicht deterministisch kontextfrei ist.

\subsection{Entscheidbarkeit}
Seien $L$ und $W$ (Turing)entscheidbar, so ist auch $L\ff W$ entscheidbar.\\\\
Seien die Turing Maschinen $T_L$ und $T_W$.\\
Die Turing Maschine $T$ entscheidet $L\ff W$ nach folgendem Algorithmus:\\

\begin{algorithm}
\caption{entscheide $L\ff W$, Input $w$}
\label{split}
\begin{algorithmic}
%\REQUIRE Menge X, int stufe
%Input: w
\IF{($w \in W \wedge w \in L$)}
\STATE $T$ accepts
\ENDIF
\STATE $w' = w$
\IF{($w \in W$)}  
\RETURN \FALSE
\ENDIF
\REPEAT
\STATE $w' \leftarrow \mathit{cut}(w')$
\IF{($w'\in W$)}
\STATE $T$ rejects
\ENDIF
\IF{($w' \in L$)}
\STATE $T$ accepts
\ENDIF
\UNTIL{($w'==e$)}
\STATE $T$ rejects
\end{algorithmic}
\end{algorithm}

\newpage

\section{Schlusswort}
\section{Quellen und Literatur}


\end{document}


