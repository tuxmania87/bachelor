
\documentclass[11pt]{scrartcl}
\usepackage{geometry}
\usepackage{graphics}
\usepackage{graphicx}
\usepackage{amssymb}
\usepackage{amstext}
\usepackage{amsmath}
\usepackage{amsthm}
\usepackage{color}
\usepackage{amsmath}
\usepackage[ngerman]{babel}
\usepackage[latin1]{inputenc}
\usepackage{units}
\usepackage{ulem}
\usepackage{algorithm}
\usepackage{algorithmic}
%\geometry{a4paper,left=20mm,right=10mm, top=13mm, bottom=20mm}
\newcommand{\ff}{\triangleright}
\author{Robert Hartmann}
\title{�ber die Operation Fortsetzung bei formalen Sprachen}
\date{24. September 2010}
\parindent 0pt
\begin{document}

\newtheorem{def1}{Definition} 
\newtheorem{satz}{Satz}
\newtheorem{lem}{Lemma}  
\newtheorem{bew}{Beweis} 

\maketitle
\newpage
\tableofcontents
\newpage
\section{Einleitung}
%einleitung
In dieser Arbeit wird die Operation Fortsetzung bei formalen Sprachen untersucht.
Im Folgenden wird erl�utert warum die Operation in [St87] eingef�hrt wurde und welchen Nutzen sie birgt.\\\\
Wir bezeichnen die Menge $X^*$ als Menge aller endlichen W�rter �ber dem Alphabet $X$. 
\\Wir bezeichnen weiterhin die Menge $X^\omega$ als Menge aller unendlichen W�rter �ber dem Alphabet $X$.
\\Sei ferner die Relation $\sqsubseteq$ wie �blicherweise definiert:
\begin{def1}
$$w\sqsubseteq b \Leftrightarrow w\cdot b' = b,\textit{ f{\"u}r ein }b'\in X^*$$
$$\pref(L) = \{v:v\sqsubseteq w \wedge w\in L\}$$
\end{def1}
Sei nun $W ^\delta$ definiert \footnote{St87]}
\begin{def1}
$$W^\delta = \{\beta : \beta \in X^\omega \textit{ und }\pref(\beta)\cap W \textit{ ist unendlich}\}$$
\end{def1}
% also eine unendliche Aneinanderreihung von W�rtern aus ....
Folgende Eigenschaft wurde nun bereits in \footnote{[St87, Gleichung 13]} bewiesen:
$$(U \cup W)^\delta = U^\delta \cup W^\delta $$
	\begin{def1}
	Eine Sprache nennen wir eine $(\sigma,\delta)$-Teilmenge von $X^*$ genau dann, wenn f�r alle $\beta \in X^\omega$ entweder $\pref(\beta)\cap W$ oder $\pref(\beta)\backslash W$ endlich ist.	
	\end{def1}
Beispiele f�r $(\sigma,\delta)$-Teilmengen sind alle endlichen Sprachen und deren Komplemente. Weitere Beispiele sind Sprachen der Form $\pref(U)$ oder $W\cdot X^*$.\\
Eine Eigenschaft f�r diese Teilmengen lautet wiefolgt:
\begin{satz}
$\text{Sei }U\text{ eine }(\sigma,\delta)-\text{Teilmenge von }X^*, \text{ dann gilt:}$\\
$$(U\cap W)^\delta = U^\delta \cap W^\delta,\qquad \text{f�r alle }W\subseteq X^*$$
\end{satz}

Nun wird die Operation "'Fortsetzung"' eingef�hrt \footnote{[St87, S.170]}, im nachfolgenden als $\ff$ bezeichnet.
Die Fortsetzung eines Wortes $w$ in $V$ sei definiert als:
\begin{def1}
$$w\ff V := \min_{\sqsubseteq} \{v:v\in V \wedge w\sqsubseteq v\} = \min(w\cdot X^* \cap V)$$
\end{def1}
Man kann diese Definition nun ausdehnen auf Sprachen. Die Fortsetzung zweier Sprachen W und V ergibt sich somit zu:
\begin{def1}
$$W\ff V := \bigcup_{w\in W} w\ff V$$
\end{def1}
Diese Operation hat nun folgende interessante Eigenschaft bez�glich der oben genannten Definitionen:
W�hrend $$(W\cap U)^\delta = W^\delta \cap U^\delta$$ nur f�r $(\sigma,\delta)$-Teilemgen gilt, so gilt aber
$$(W\ff U)^\delta = W^\delta \cap U^\delta$$ f�r s�mtliche Sprachen \footnote{[St87, Gleichung 20]}\\
Daher wird nun im Verlauf der Arbeit die Operation Fortsetzung gr�ndlich analysiert und all ihre Eigenschaften dokumentiert.


\section{allgemeine Eigenschaften}
\subsection{Gilt f�r alle Sprachen}

Folgende Eigenschaft ist direkt aus der Definition einsehbar:
\begin{gleich}
$$u\in L \to (u\ff L=\{u\})$$
\end{gleich}

Aus 2.1.1 folgt direkt
\begin{gleich}
$$L\ff L = L$$
\end{gleich}
\begin{gleich}
$$U\ff L \subseteq L$$
\end{gleich}

Eine unmittelbare Folgerung aus der Definition 1.5 ergibt sich:
\begin{gleich}
$$(L\cup U)\ff V = (L\ff V) \cup (U\ff V)$$
\end{gleich}
\begin{proof}
$$(L\cup U) \ff V = \bigcup_{l\in L\cup U} \min(l\cdot X^* \cap V)$$
$$ =\bigcup_{l\in L} \min(l\cdot X^* \cap V) \cup \bigcup_{l\in U} \min(l\cdot X^* \cap V)$$
$$ = (L\ff V) \cup (U\ff V)$$
\end{proof}

Aus dieser Gleichung 2.1.4 folgt wiederum direkt:

\begin{gleich}
$$L_2\subseteq L_1 \to L_2\ff W \subseteq L_1 \ff W$$
\end{gleich}
\begin{proof}
$$L_1\ff W = \bigcup_{l\in L_1} l\ff W$$
$$=\bigcup_{l\in L_2} l\ff W \cup \bigcup_{l\in L_1\backslash L_2} l\ff W$$
$$\supseteq \bigcup_{l\in L_2} l\ff W = L_2\ff W$$
\end{proof}

Auf Gleichung 2.1.5 folgt direkt:

\begin{gleich}
$$(L\cap U) \ff V \subseteq (L\ff V) \cap (U\ff V)$$
\end{gleich}
\begin{proof}
$$w\in  (L\cap U)\ff V \to w\in V \wedge \exists p: p\sqsubseteq w \wedge p\in L \wedge p\in U \wedge p\textit{ ist minimal}$$
$$\to \underbrace{w\in V \wedge \exists p: p\sqsubseteq w \wedge p\in L \wedge p\textit{ ist minimal}}_{w\in L\ff V} 
\wedge \underbrace{w\in V \wedge \exists p: p\sqsubseteq w \wedge p\in U \wedge p\textit{ ist minimal}}_{w\in U\ff V} 
$$
\end{proof}


\begin{gleich}
$$L\ff (U\cup V)\subseteq (L\ff U) \cup (L\ff V)$$
\end{gleich}
\begin{lem}
$$\min (A\cup B) \subseteq \min A \cup \min B$$
\end{lem}
Es gen�gt, die Eigenschaft f�r $L=\{w\}$ zu zeigen.
\begin{proof}
$$L\ff (W\cup V)= w \ff (W\cup V)$$
$$=\min (w\cdot X^* \cap (W\cup V))$$
Nach Anwenden der Distributivgesetze ergibt sicht:
$$=\min ((w\cdot X^* \cap W) \cup (w\cdot X^* \cap V))$$
Anwendung von Lemma 1:
$$\subseteq \min (w\cdot X^* \cap W) \cup \min (w\cdot X^* \cap V)$$
$$=(L\ff W)\cup (L\ff V)$$
\end{proof}

\begin{gleich}
$$L\ff (U\cap V)\supseteq (L\ff U) \cap (L\ff V)$$
\end{gleich}
\begin{lem}
$$\min(A\cap B) \supseteq \min A \cap \min B$$
\end{lem}
Es gen�gt die Eigenschaft f�r $L=\{w\}$ zu zeigen.
%\begin{lem}
%$$L\ff U \cap L \ff V \stackrel{!}{=} \bigcup_{l\in L} (l\ff U \cap l\ff V)$$
%\end{lem}
\begin{proof}
$$L\ff (U\cap V) = \min(w\cdot X^* \cap (U \cap V))$$
Nach Anwenden der Distributivgesetze ergibt sicht:
$$= \min((w\cdot X^* \cap U) \cap (w\cdot X^* \cap V))$$
Anwendung von Lemma 2:
$$\supseteq \min(w\cdot X^* \cap U) \cap \min(w\cdot X^* \cap V)$$
$$=(L\ff U) \cap (L\ff V)$$
\end{proof}

\begin{gleich}
$$L_1\subseteq L_2 \to L_1\ff L_2 = L_1$$
\end{gleich}
\begin{eigen}
$$l\in L \to l\ff L = \{l\}$$
\end{eigen}
\begin{proof}
$$L_1\ff L_2 = \bigcup_{l\in L_1} l\ff L_2$$
$$\textit{wegen } L_1\subseteq L_2: l\ff L_2 = \{l\}$$
$$ = \bigcup_{l\in L_1} \{l\} = L_1$$
\end{proof}

\begin{gleich}
$$L_2\subseteq L_1 \to L_1\ff L_2 = L_2$$
\end{gleich}
\begin{proof}
$$\textit{Da }L_2\subseteq L_1\textit{ gilt f�r alle }l\in L_2 \wedge l\in L_1: l\ff L_2=\{l\}$$
$$\bigcup_{l\in L_1} l\ff L_2 = \bigcup_{l\in L_1} \{l\} = L_2$$
\end{proof}

\subsection{Gilt f�r einige Sprachen $\exists L,U,V \subseteq X^*:$}
$$L\ff (U\cdot V) = (L\ff U) \cdot (L\ff V)$$
$$L\ff (U\cdot V) \supset (L\ff U) \cdot (L\ff V)$$


\subsection{Gilt nicht...}
Folgt direkt aus den Gleichungen in 2.1
$$L\ff (U\cup V)\supset (L\ff U) \cup (L\ff V)$$
$$L\ff (U\cap V)\subset (L\ff U) \cap (L\ff V)$$
$$L\ff (U\cdot V)\subset (L\ff U) \cdot (L\ff V)$$
$$(L\cap U) \ff V \supset (L\ff V) \cap (U\ff V)$$

\newpage
$$L\ff (U\cup V)\subset (L\ff U) \cup (L\ff V)$$
$$L= \{a,b\} U =\{aaa\} V = \{bb,aa\}$$\\
$$L\ff (U\cup V) = (L\ff U) \cup (L\ff V)$$
$$L= \{a,b\} U =\{aaa\} V = \{aaa\}$$\\
$$L\ff (U\cap V)\supset (L\ff U) \cap (L\ff V)$$
$$L= \{a,b\} U =\{aaa,b,bb\} V = \{bb,aaa\}$$\\
$$L\ff (U\cap V)= (L\ff U) \cap (L\ff V)$$
$$L= \{a\} U =\{aaa\} V = \{aaa\}$$\\
$$(L\cap U) \ff V \subset (L\ff V) \cap (U\ff V)$$
$$L= \{aa,bb\} U = \{aa,b\} V = \{aa,bb\}$$\\
$$(L\cap U) \ff V = (L\ff V) \cap (U\ff V)$$
$$L= \{aaa\} U =\{aaa\} V = \{aaa\}$$\\
\vspace{5mm}
$$L\ff (U\cup V) \not\supset (L\ff U) \cup (L\ff V),$$
$$\text{Gegenbeispiel: } L=\{a\}\quad U=\{abb,aaba\}\quad V=\{aab,aba\} $$
$$L\ff (U\cup V) = \{abb,aba,aab\}\text{ ,aber } L\ff U \cup L\ff V = \{abb,aaba\} \cup \{aab,aba\} = \{abb,aaba,aab,aba\} $$
\vspace{5mm}
$$L\ff (U\cap V) \not\subset (L\ff U) \cap (L\ff V),$$
$$\text{Gegenbeispiel: } L=\{a,b\}\quad U=\{a,aa\}\quad V=\{aa,b\} $$
$$L\ff (U\cap V) = \{aa\}\text{ ,aber } L\ff U \cap L\ff V =
\{a\} \cap \{b\} = \emptyset $$
\vspace{5mm}
$$L\ff (U\cdot V)= (L\ff U) \cdot (L\ff V),$$
$$\text{Beispiel: } L=\{e\}\quad U=\{a\}\quad V=\{b\} $$
$$L\ff (U\cdot V) \supset (L\ff U) \cdot (L\ff V),$$
$$\text{Beispiel: } L=\{a\}\quad U=\{aa\}\quad V=\{b,a\} $$
\vspace{5mm}
$$(L_1 \ff L_2) \ff L_3 \nsupseteq L_1 \ff (L_2 \ff L_3)$$
$$\text{Gegenbeispiel: } L_1=\{ab,aa\}\quad L_2=\{a,ab\}\quad L_3=\{aa\}$$
$$(L_1 \ff L_2) \ff L_3 = \{ab\} \ff \{aa\} = \emptyset\text{ ,aber } 
\{ab,aa\} \ff \{aa\} = \{aa\} $$

\newpage

\section{Eigenschaften bei Sprachen spezieller Gestalt}
%\begin{eigen}
%$$V\ff W\cdot X^* = V\ff W $$
%\end{eigen}

%Es gen�gt die Eigenschaft zu zeigen f�r $L=\{v\}$:
%\begin{proof}
%$$V\ff W\cdot X^* = \{v\} \ff W\cdot X^*$$
%Nach Definition der Operation Fortsetzung ergibt sich:
%$$ = min_{\sqsubseteq} \{w:w\in W\cdot X^* \wedge v\sqsubseteq w\}$$
%Wir wissen aus der Vorrausetzung, dass $v\notin W\cdot X^*$, daraus folgt unmittelbar, dass $v\in pref(W)\backslash W$.\\
%Angenommen es existiert ein $w\in W\cdot X^+$ in $min_{\sqsubseteq} \{w:w\in W\cdot X^* \wedge v\sqsubseteq w\}$, so ist $w'$ mit $w=w'\cdot r\wedge w'\in W$ ein k�rzeres Wort, daher gilt:
%$$min_{\sqsubseteq} \{w:w\in W\cdot X^* \wedge v\sqsubseteq w\} = min_{\sqsubseteq} \{w:w\in W \wedge v\sqsubseteq w\}$$
%\end{proof}

%\begin{eigen}
%$$V\ff W = V\ff \min(W)$$
%\end{eigen}

\begin{lem}\label{eingeschr1}
Sei $V\subseteq X^*\backslash W\cdot X^*$, so gilt $V\ff W\cdot X^* = V\ff \min(W) $
\end{lem}
z.z:\\\\
1. $V\ff W\cdot X^* \subseteq V\ff \min(W) $\\
2. $V\ff W\cdot X^* \supseteq V\ff \min(W) $\\

\textbf{zu 1.} Es gen�gt die Eigenschaft zu zeigen f�r $V=\{v\}$\\
\begin{proof}
$$v \ff W\cdot X^* = \min_{\sqsubseteq} \{w:w\in W\cdot X^* \wedge v\sqsubseteq w\}$$
$$w\in W\cdot X^* \wedge v\sqsubseteq w \wedge v\not\in W\cdot X^* \to ( w'\in (\min(v\cdot X^* \cap W\cdot X^*) \to w'\in  (\min(v\cdot X^* \cap \min(W)))) )$$
\end{proof}

\textbf{zu 2.} klar, weil $W\cdot X^* \supseteq \min(W)$



\newpage

\begin{eigen}
$\pref(V)\ff W$
\end{eigen}
\begin{proof}
1. Fall: $w\in W \cap \pref(V) \to w\in \pref(V)\ff W$\\
2. Fall: $w\in W\backslash \pref(V) \to w\in V\ff Min(W)$\\
$\to \pref(V)\ff W = (\pref(V)\cap W) \cup (\pref(V)\ff \min(W))$\\\\
\end{proof}

\begin{eigen}
$W\ff \pref(V) = W \cap \pref(V)$
\end{eigen}
Es gen�gt die Eigenschaft f�r $W=\{w\}$ zu zeigen:
\begin{proof}
$$W\ff \pref(V) = \{w\}\ff\pref(V)$$
$$=min_{\sqsubseteq} \{v:v\in \pref(V) \wedge w\sqsubseteq v\}$$
Aus $v\in \pref(V) \wedge w\sqsubseteq v$ folgt direkt, dass $w\in pref(V)$. Damit ergibt sich :
$$min_{\sqsubseteq} \{v:v\in \pref(V) \wedge w\sqsubseteq v\} = \{v:v\in\pref(V)\wedge w=v\} = \{w\} \cap \pref(V)$$
\end{proof}
%"'$\leftarrow$"'\\
%$w\in \pref(V) \wedge w\in W \to w\sqsubseteq v\in \pref(V)$\\\\
\begin{eigen}
$V\cdot X^* \ff W$
\end{eigen}
Es gen�gt die Eigenschaft f�r ein $v\in V\cdot X^*$ zu zeigen:
\begin{proof}
$$\min_{\sqsubseteq} \{w:w\in W \wedge v\sqsubseteq w\}$$
$$v\in VX^* \wedge v\sqsubseteq w \to w\in VX^*$$
$$=\min_{\sqsubseteq} \{w:w\in W \wedge w\in V\cdot X^* \wedge v=w\} = \{v\}\cap W$$
%$=\bigcup_{v\in V} \{w:w\in W \wedge v\sqsubseteq w\}$\\\\
\end{proof}

\begin{eigen}
$$W\ff V\cdot X^*$$
\end{eigen}
\begin{proof}
$W$ l�sst sich in 2 Teile aufspliten: $W= (W\cap V\cdot X^*) \cup (W\backslash V\cdot X^*)$
Nun betrachten wir folgende 2 F�lle:\\\\
Fall a) $w\in W\cap V\cdot X^*$\\
Fall b) $w\in W\backslash V\cdot X^*$\\\\
zu a): $w\ff V\cdot X^* \to w\in (W\cap V\cdot X^*)$\\
zu b): Es gilt nach Vorraussetzung $\{w\} \subseteq X^*\backslash V\cdot X^*$ und mit Hilfe von Lemma~\ref{eingeschr1}
erhalten wir $\{w\} \ff V\cdot X^* = \{w\} \ff \min(V)$
\\Daraus folgt: $$W\ff V\cdot X^* = (W\ff\min(V)) \cup (W\cap V\cdot X^*)$$
\end{proof}





%\newpage
%$$ V = \{v:\forall w(w\in W \to w\not\sqsubseteq v)\}\qquad \mathit{pref}(V)=\{u:\exists v (u\sqsubseteq v\wedge v\in V)\}$$\\
%$$W\ff V = \emptyset$$
%$$W\ff \mathit{pref}(V) = \emptyset$$
%$$W\cdot X^* \ff V = \emptyset$$
%$$W\cdot X^* \ff \mathit{pref}(V) = \emptyset$$
%$$\mathit{pref}(V)\ff W\cdot X^* = $$

%$$W\ff V = \bigcup_{x\in W} x\ff V = \bigcup \min_{\sqsubseteq } \{v:v\in V \wedge x\sqsubseteq v\}$$
%Wegen $x\in W$ und $x\sqsubseteq v$ gilt $\{v:v\in V \wedge x\sqsubseteq v\} = \emptyset \Rightarrow \bigcup \min_{\sqsubseteq} \{v:v\in V \wedge x\sqsubseteq v\} = \emptyset$\\\\
%$$ W\cdot X^* \ff V = \bigcup_{x\in W\cdot X^*} x\ff V = \bigcup_{x\in W\cdot X^*} \min_{\sqsubseteq} \{v:v\in V \wedge x\sqsubseteq v\} = \emptyset$$
%Begr�ndung analog Fall $W\ff V$\\\\
%$$ W \ff \mathit{pref}(V) = \bigcup_{w\in W} w\ff \mathit{pref}(V) = \bigcup_{w\in W} \{v:v\in \mathit{pref}(V) \wedge w\sqsubseteq v\} = \emptyset$$
%$w\sqsubseteq v$ ist nie erfuellt. Angenommen $v\in \mathit{pref}(V) \wedge w\sqsubseteq v$, dann kann man $v$ wie folgt verlaengern $v'=v\cdot v''$ mit $v'\in V$, dann waere $v'=w\cdot w' \cdot v''$ mit $w\cdot w' = v$.
%Dadurch gilt aber $v'\in V\wedge w\sqsubseteq v' \wedge w\in W \Rightarrow$ Widerspruch zur Definition.\\\\

%$$\mathit{pref}(V) \ff W\cdot X^* = \bigcup_{u\in \mathit{pref}(V)} u\ff W\cdot X^* = \bigcup_{u\in \mathit{pref}(V)} \min_{\sqsubseteq} \{w:w\in W\cdot X^* \wedge u\sqsubseteq w\}$$
%$$ =  \bigcup_{u\in \mathit{pref}(V)} \min(W\cdot X^* \cap u\cdot X^*) =  \bigcup_{u\in \mathit{pref}(V)} \min(\{w:w\in W \wedge u\sqsubseteq w\})$$\\
%$$W\cdot X^* \ff \mathit{pref}(V) = \bigcup_{u\in W\cdot X^*} u\ff \mathit{pref}(V) = \bigcup_{u\in W\cdot X^*} \min_{\sqsubseteq} \{w:w\in \mathit{pref}(V) \wedge u\sqsubseteq w\}$$
%$$ = \bigcup_{u\in W\cdot X^*} \min(\mathit{pref}(V)\cap u\cdot X^*)$$
%$\mathit{pref}(V)\cap u\cdot X^* = \emptyset$, Begr�ndung siehe oben $\Rightarrow  \bigcup_{u\in W\cdot X^*} \min(\emptyset) = \emptyset$

\newpage

\section{Abgeschlossenheit in der CHOMSKY-Hierachie}
In diesem Abschnitt untersuchen wir die Abgeschlossenheitseigenschaften der Klassen der CHOMSKY-Hierachie bez"uglich der Operation Fortsetzung. 
Dabei betrachten wir in den folgenden Abschnitten die Klassen der regul"aren, kontextfreien, entscheidbaren und aufz"ahlbaren Sprachen.

\section{Regularit"at}\label{abschnittreg}
\begin{satz}
Sind $L$ und $W$ regul\"are Sprachen, so ist auch $L\ff W$ regul\"ar.
\end{satz}
\begin{proof}
Der nichtdeterministische Automat $A_L= (X,Z,z_{0},\delta_L,Z_f)$ akzeptiere $L$, der deterministische Automat $A_W = (X,S,s_{0},f,S_{f})$ akzeptiere $W$.
Wir konstruieren einen nichtdeterministischen Automaten $A$, der $L\ff W$ akzeptiert.\\\\\emph{Arbeitsweise.}\\
In Phase eins lesen $A_L$ und $A_W$ das Wort $w$ parallel (\ref{auto1}). Falls $A_L$ ein Pr"afix $v$ von $w$ akzeptiert, w"ahlt $A$ nichtdeterministisch aus, ob Schritt zwei aktiviert wird oder nicht ( Nichtdeterminismus von $A_L$ ).\\
Wird Phase zwei aktiviert, so liest $A_W$ das Wort $w$ weiter, w"ahrend $A_L$ in einem Ruhezustand $z_f'$ verweilt (\ref{auto2}). Sobald $A_W$ in Phase zwei akzeptiert, gelangt $A$ nach Konstruktion in den Zustand $(z_f',s_x)$, aus dem keine Transition herausf"uhrt (\ref{auto4}). Akzeptiert $A_W$ nun ein Pr"afix $v'$ mit $v\sqsubseteq v'\sqsubset w\text{ und } v'\in W$, so muss $A$ das Eingabewort $w$ ablehnen. Dies wird realisiert, da keine Transition aus $(z_f',s_x)$ herausf"uhrt und $w$ noch nicht zu Ende gelesen ist. Findet der Automat $A_W$ kein solches Pr"afix $v'$ und akzeptiert das Eingabewort $w$ ( und kein Pr"afix von $w$), so akzeptiert auch $A$, weil $w$ zu Ende gelesen wurde und sich $A$ im Finalzustand $(z_f',s_x)$ (\ref{auto4}) befindet.
\\$A = (X,(Z\cup \{ z_f'\})\times (S\cup \{ s_x\}), (z_{0},s_{0}), \delta , \{ (z_f',s_x) \} ),$ mit $s_x\notin S,\ z_f'\notin Z$
\setcounter{equation}{0}
\begin{eqnarray}
 \delta &=&\{ ((z_i,s_i),x,(z_j,s_j)) : (z_i,x,z_j) \in \delta_L \wedge f(s_i,x)=s_j  \}  \label{auto1} \\
 & \cup & \{ ((z_i,s_i),x,(z_f',s_j)) : (z_i,x,z') \in \delta_L \wedge z'\in Z_f \wedge f(s_i,x)=s_j  \} \label{auto2}\\
 & \cup & \{ ((z_f',s_i),x,(z_f',s_j)) : f(s_i,x)=s_j  \wedge s_j \notin S_f\} \label{auto3}\\
 & \cup & \{ ((z_f',s_i),x,(z_f',s_x)) : f(s_i,x)=s_j  \wedge s_j \in S_f \label{auto4} \}
\end{eqnarray}
Nach Konstruktion ist klar, dass der Automat $A$ nur W"orter aus $L\ff W$ akzeptiert. Falls $w\in L\ff W$, so gibt es ein Präfix $v\in L$ derart, dass $w\in \{v\}\ff W$. Weil dieses Pr"afix nichtdeterministisch von $A_L$ ausgew"ahlt wird, akzeptiert $A$ auch alle W"orter aus $L\ff W$.
Damit akzeptiert der Automat $A$ ein Eingabewort $w$ genau dann, wenn $w\in L\ff W$.
\end{proof}

\section{Kontextfreiheit}
Es existieren deterministisch kontextfreie, lineare Sprachen $L$ und $W$ derart, dass $L\ff W$ nicht einmal kontextfrei ist.

\vspace{2ex}

\begin{beispiel}
Es seien $L=\{a^nb^nc^i:i,n>0\}$ und $W=\{a^ib^nc^n:i,n>0\}$. Sowohl $L$ als auch $W$ sind deterministische, kontextfreie und auch lineare Sprachen, da es je einen deterministischen Kellerautomaten gibt, der $L$ sowie $W$ akzeptiert und es lineare Grammatiken \\\\$G_L = (\{S,A,C\},\{a,b,c\},S,\{ (S,Cc),(C,Cc),(C,aAb),(A,aAb),(A,e)\})$ und \\$G_W = (\{S,A,B\},\{a,b,c\},S,\{ (S,aA),(A,aA),(A,bBc),(B,bBc),(B,e) \})$ derart gibt, dass \\$L(G_L) = L$ und $L(G_W) = W$ gilt.\\\\
Betrachten wir nun ein Wort $u\in L\ff W$, so muss $u$ laut Definition folgende Struktur besitzen: $u\in \min( v\cdot X^* \cap W)$ f"ur ein $v \in L$.
Damit sieht man leicht, dass $L\ff W= \{a^nb^nc^n:n>0\}$ gilt. Diese Sprache ist bekanntlich nicht einmal kontextfrei.
\end{beispiel}

\vspace{2ex}

\begin{satz}
Ist $L$ eine kontextfreie Sprache und $W$ eine reguläre Sprache, so ist $L\ff W$ ebenfalls kontextfrei.
\end{satz}
Zum Beweis k"onnten wir die Konstruktion aus Abschnitt \emph{\ref{abschnittreg} Regularit"at} verwenden, indem wir den nichtdeterministischen endlichen Automaten $A_L$ durch einen nichtdeterministischen Kellerautomaten ersetzen.
Ein anderer Beweis f"ur diesen Satz wurde in [St97, Abschnitt \emph{Limit-closure}] gef"uhrt.
\newpage
\section{Entscheidbarkeit und Aufz"ahlbarkeit}
In diesem Abschnitt untersuchen wir, ob f"ur entscheidbare bzw. aufz"ahlbare Sprachen $L$ und $W$ auch deren Fortsetzung $L\ff W$ wiederum entscheidbar bzw. aufz"ahlbar ist. Dabei betrachten wir drei verschieden F"alle.\\\\
Wir untersuchen zun"achst $L\ff W$, wenn $L$ und $W$ entscheidbar sind. Dabei werden wir feststellen, dass auch die Fortsetzung $L\ff W$ entscheidbar ist. Ein Algorithmus wird dazu angegeben.\\
Anschlie"send werden wir die Fortsetzung betrachten, wenn $L$ nur aufz"ahlbar und $W$ entscheidbar ist. In diesem Fall ist die Fortsetzung von $L$ in $W$ ebenso aufz"ahlbar. Dazu geben wir einen Algorithmus an.\\
Im letzten Fall betrachten wir $L\ff W$, wenn $W$ nur aufz"ahlbar ist. In diesem Fall ist die Fortsetzung beider Sprachen nicht einmal aufz"ahlbar. Dies gilt sogar, wenn $L$ eine endliche Sprache ist.
\subsection{L und W entscheidbar}

\begin{satz}\label{entsch1}
Sind $L$ und $W$ entscheidbare Sprachen, so ist auch $L\ff W$ entscheidbar.
\end{satz}
\begin{proof}
Wir wissen, dass zu jeder entscheidbaren Sprache $L$ ein Algorithmus angegeben werden kann, welcher $L$ entscheidet. Daher geben wir zum Beweis von Satz \ref{entsch1} einen Algorithmus an, welcher $L\ff W$ entscheidet (befindet sich auf n"achster Seite).\\\\
\emph{Arbeitsweise des Algorithmus:}\\
Zun"achst pr"ufen wir, ob $w\in W$, da dies eine zwingende Voraussetzung nach Definition der Fortsetzung ist (\# 1).
Ist $w\notin W$, so lehnen wir ab. Ist $w\in W$ und zus"atzlich $w\in L$, so ist klar, dass $w\in L\ff W$ nach Folgerung \ref{klaro} (\# 2).\\\\
Jetzt setzen wir $w'$ mit $w':=w$ als Arbeitskopie. Im folgenden Schleifendurchlauf schneiden wir mit $cut()$ den letzten Buchstaben von $w'$ ab (\# 3). 
Gilt jetzt $w'\in W$ so m"ussen wir mit NEIN entscheiden, da das Präfix $w'\in W$ im Widerspruch zum $\min$ Kriterium steht.
Anschlie"send pr"ufen wir, ob $w'\in L\setminus W$ (\# 5) gilt. Wenn ja, dann haben wir das erste Pr"afix von $w$ gefunden, welches nicht in $W$ liegt, und k"onnen f"ur die Eingabe $w$ mit JA entscheiden.
\\\\Ansonsten durchlaufen wir die Schleife solange weiter, bis $w'$ nur noch aus dem leeren Wort besteht, anschliessend lehnen wir ab, da wir kein Pr"afix $w'$ von $w$ gefunden haben, welches in $L\setminus W$ liegt.
\begin{algorithm}
\caption{entscheide $L\ff W$}
\label{split}
\begin{algorithmic}
%\REQUIRE Menge X, int stufe
%Input: w
\STATE Input $w$
\IF{($w \notin W$)}  
\STATE $T$ rejects \COMMENT{\# 1}
\ELSE
\IF{($w \in L$)}
\STATE $T$ accepts  \COMMENT{\# 2}
\ENDIF
\ENDIF
\STATE $w' = w$  \COMMENT{\# 3}
\REPEAT
\STATE $w' \leftarrow \mathit{cut}(w')$ \COMMENT{\# 4}
\IF{($w'\in W$)}
\STATE $T$ rejects
\ENDIF
\COMMENT{\# 5}
\IF{($w' \in L$)}
\STATE $T$ accepts
\ENDIF
\UNTIL{($w'==e$)}
\STATE $T$ rejects
\end{algorithmic}
\end{algorithm}

\end{proof}


\subsection{L aufz"ahlbar, W entscheidbar}
\begin{satz}
Ist $L$ eine aufz"ahlbare Sprache und $W$ eine entscheidbare Sprache, so ist $L\ff W$ aufz"ahlbar.
\end{satz}
\begin{proof}
Es sei $U=\{u_i : 0 \le i \le n \wedge u_i \in L\}$ eine Aufz"ahlung von $L$.\\
Wir konstruieren einen Algorithmus, welcher die Eingabe $w$ akzeptiert genau dann, wenn $w\in L\ff W$.
Ist $w\notin W$ so kann nach Definition auch nicht $w\in L\ff W$ gelten. \\\\Wir w"ahlen das Wort $u_i$ aus der Aufz"ahlung $U$ von $L$ aus (\# 1).
Gilt nun $u_i\sqsubseteq w$, so wird eine Arbeitskopie $v$ mit $v:=w$ erstellt (\# 2). Im n"achsten Schritt pr"ufen wir f"ur alle Pr"afixe $v$ mit $u_i\sqsubseteq v\sqsubset w$ sukzessive, ob $v\in W$ (\# 3). 
\\\\Sollte dies der Fall sein, dann haben wir mit $v$ ein Pr"afix von $w$, welches gegen das $\min$ Kriterium verst"o"st. Dann verlassen wir die innere Schleife und wählen in der FOR-Schleife das n"achste $u_{i+1}$ aus(\# 4) und der Algorithmus l"auft \emph{ad infinitum}.% und der Algorithmus l"auft mit diesem $u_i$ weiter.
\\Gilt aber f"ur alle Pr"afixe $v\notin W$, so erreicht die innere Schleife ihre Abbruchbedingung und wir akzeptieren die Eingabe $w$ (\# 5).


\begin{algorithm}
\caption{akzeptiere $L\ff W$}
\label{split2}
\begin{algorithmic}
%\REQUIRE Menge X, int stufe
%Input: w
\STATE Input $w$
\STATE Sei
\IF{($w \notin W$)}  
\STATE $T$ rejects
\ENDIF\\
\COMMENT{\# 1}
\FOR{$i = 0$ to $\infty$} 
%\STATE z"ahle ein $u\in L$ auf 
\IF{  $u_i\in\pref(\{w\})$ }
\STATE $v := w$ \COMMENT{\# 2}
\WHILE{$v\neq u_i$}
\STATE $v \leftarrow cut(v)$ \COMMENT{\# 3}
\STATE \textbf{EXIT IF} $v\in W$ \COMMENT{\# 4}
\ENDWHILE
\IF{$v=u_i$}
\STATE $T$ accepts \COMMENT{\# 5}
\ENDIF
\ENDIF
\ENDFOR

\end{algorithmic}
\end{algorithm}
\end{proof}
\newpage
\subsection{L entscheidbar, W aufz"ahlbar}

\begin{satz}
Ist $L$ eine entscheidbare Sprache und $W$ eine aufz"ahlbare Sprache, so ist $L\ff W$ nicht notwendig aufz"ahlbar.
\end{satz}
\begin{proof}
Wir w"ahlen ein $A\subseteq \mathbb{N}$, welches aufz"ahlbar, aber nicht entscheidbar ist und
setzen $W$ wie in (\ref{last1}) und $L$ wie in (\ref{last2}). $W$ ist also aufz"ahlbar und $L$ sogar regul"ar und endlich.
Angenommen (\ref{last3}) w"are aufz"ahlbar, so w"are (\ref{last4}) auch aufz"ahlbar. Das w"urde bedeuten, dass $\mathbb{N}\setminus A$ aufz"ahlbar und damit $A$ entscheidbar w"are. Dies ergibt aber einen Widerspruch zur Annahme, da wir $A$ so gew"ahlt haben, dass es nicht entscheidbar ist.
\setcounter{equation}{0}
\begin{eqnarray}
W&=&\{0^{n+1}\ 1^{n+1} : n\in \mathbb{N} \} \cup \{0^{n+1}\ 1 : n\in A \} \label{last1} \\
L&=&\{0\} \label{last2} \\
L\ff W = \{0^{n+1}\ 1^{n+1} &:& n\in\mathbb{N} \wedge n\notin A\} \cup \{0^{n+1}\ 1:n\in \mathbb{N} \wedge n\in A\} \label{last3} \\
(L\ff W)\cap \{0^n\ 1^n : n\in \mathbb{N}\} &=& \{0^{n+1}\ 1^{n+1} : n\in \mathbb{N} \wedge n\notin A\} \label{last4} 
\end{eqnarray}
\end{proof}

\newpage

\section{Schlusswort}
\section{Quellen und Literatur}


\end{document}


